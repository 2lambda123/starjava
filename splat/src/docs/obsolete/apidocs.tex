% ? Document specific \newcommand or \newenvironment commands.

% LaTeX doclet defintions.

%begin{latexonly}
\newcommand{\entityintro}[3]{
  \textbf{#1}
  \dotfill\pageref{#2}
  \begin{quote}
  #3
  \end{quote}
}

\newcommand{\refdefined}[1]{}

\newcommand{\startsection}[4]{
   \subsubsection{\label{#3}{#2}}
   #4
}

\newcommand{\startsubsubsection}[1]{
   \subsubsection{#1}
   \hrule
}

\newcommand{\method}[1]{\texttt{#1}}

\newenvironment{desc}{\begin{quote}}{\end{quote}}

\newcommand{\constructors}{
   \subsubsection*{Constructors}
   \hrule
}

\newcommand{\methods}{
   \subsubsection*{Methods}
   \hrule
}

\newcommand{\inherited}[1]{
   \subsubsection*{Methods inherited from class #1}
   \hrule
}

\newcommand{\fields}[1]{
   \subsubsection*{#1}
   \hrule
}

\newcommand{\field}[2]{
   \texttt{#1} 
   \begin{quote}
   #2
   \end{quote}
}
%end{latexonly}
\begin{htmlonly}

\newcommand{\entityintro}[3]{
  \htmlref{\textbf{\Large{#1}}}{#2}
  \begin{quote}
  #3
  \end{quote}
}

\newcommand{\refdefined}[1]{}

\newcommand{\startsection}[4]{
   \subsubsection{\label{#3}{#2}}
   #4
}

\newcommand{\startsubsubsection}[1]{
   \begin{quote}
   \textbf{\large{#1}}
   \end{quote}
}

\newcommand{\method}[1]{\texttt{#1}}

\newenvironment{desc}{\begin{quote}}{\end{quote}}

\newcommand{\constructors}{
   \par\textbf{\large{Constructors}}\\
   \hrule
}

\newcommand{\methods}{
   \par\textbf{\large{Methods}}\\
   \hrule
}

\newcommand{\inherited}[1]{
   \par\textbf{\large{Methods inherited from class #1}}\\
   \hrule
}

\newcommand{\fields}[1]{
   \par\textbf{\large{#1}}\\
   \hrule
}

\newcommand{\field}[2]{
   \par\texttt{#1}
   \begin{itemize}
   \item #2
   \end{itemize}
}
\end{htmlonly}

% ? End of document specific commands
% -----------------------------------------------------------------------------
%  Title Page.
%  ===========
\renewcommand{\thepage}{\roman{page}}
\begin{document}
\thispagestyle{empty}

%  Latex document header.
%  ======================
\begin{latexonly}
   CCLRC / \textsc{Rutherford Appleton Laboratory} \hfill \textbf{\stardocname}\\
   {\large Particle Physics \& Astronomy Research Council}\\
   {\large Starlink Project\\}
   {\large \stardoccategory\ \stardocnumber}
   \begin{flushright}
   \stardocauthors\\
   \stardocdate
   \end{flushright}
   \vspace{-4mm}
   \rule{\textwidth}{0.5mm}
   \vspace{5mm}
   \begin{center}
   {\Huge\textbf{\stardoctitle \\ [2.5ex]}}
   {\LARGE\textbf{\stardocversion \\ [4ex]}}
   {\Huge\textbf{\stardocmanual}}
   \end{center}
   \vspace{5mm}

% ? Add picture here if required for the LaTeX version.
%   e.g. \includegraphics[scale=0.3]{filename.ps}
% ? End of picture

% ? Heading for abstract if used.
   \vspace{10mm}
   \begin{center}
      {\Large\textbf{Abstract}}
   \end{center}
% ? End of heading for abstract.
\end{latexonly}

%  HTML documentation header.
%  ==========================
\begin{htmlonly}
   \xlabel{}
   \begin{rawhtml} <H1> \end{rawhtml}
      \stardoctitle\\
      \stardocversion\\
      \stardocmanual
   \begin{rawhtml} </H1> <HR> \end{rawhtml}

% ? Add picture here if required for the hypertext version.
%   e.g. \includegraphics[scale=0.7]{filename.ps}
% ? End of picture

   \begin{rawhtml} <P> <I> \end{rawhtml}
   \stardoccategory\ \stardocnumber \\
   \stardocauthors \\
   \stardocdate
   \begin{rawhtml} </I> </P> <H3> \end{rawhtml}
      \htmladdnormallink{CCLRC / Rutherford Appleton Laboratory}
                        {http://www.cclrc.ac.uk} \\
      \htmladdnormallink{Particle Physics \& Astronomy Research Council}
                        {http://www.pparc.ac.uk} \\
   \begin{rawhtml} </H3> <H2> \end{rawhtml}
      \htmladdnormallink{Starlink Project}{http://www.starlink.rl.ac.uk/}
   \begin{rawhtml} </H2> \end{rawhtml}
   \htmladdnormallink{\htmladdimg{source.gif} Retrieve hardcopy}
      {http://www.starlink.rl.ac.uk/cgi-bin/hcserver?\stardocsource}\\

%  HTML document table of contents.
%  ================================
%  Add table of contents header and a navigation button to return to this
%  point in the document (this should always go before the abstract \section).
  \label{stardoccontents}
  \begin{rawhtml}
    <HR>
    <H2>Contents</H2>
  \end{rawhtml}
  \htmladdtonavigation{\htmlref{\htmladdimg{contents_motif.gif}}
        {stardoccontents}}

% ? New section for abstract if used.
  \section{\xlabel{abstract}Abstract}
% ? End of new section for abstract
\end{htmlonly}

% -----------------------------------------------------------------------------
% ? Document Abstract. (if used)
%  ==================
\stardocabstract
% ? End of document abstract

% -----------------------------------------------------------------------------
% ? Latex Copyright Statement
%  =========================
\begin{latexonly}
\newpage
\vspace*{\fill}
\stardoccopyright
\end{latexonly}
% ? End of Latex copyright statement

% -----------------------------------------------------------------------------
% ? Latex document Table of Contents (if used).
%  ===========================================
  \newpage
  \begin{latexonly}
    \setlength{\parskip}{0mm}
    \tableofcontents
    \setlength{\parskip}{\medskipamount}
    \markboth{\stardocname}{\stardocname}
  \end{latexonly}
% ? End of Latex document table of contents
% -----------------------------------------------------------------------------

\cleardoublepage
\renewcommand{\thepage}{\arabic{page}}
\setcounter{page}{1}

\section{Package uk.ac.starlink.splat.data}

\vspace{.13in}
\hbox{\textbf{Classes}}
\vspace{.13in}
\entityintro{MEMSpecDataImpl}{l0}{This class provides an implementation of SpecDataImpl to access
 spectra stored in existing memory.}
\entityintro{SpecData}{l1}{SpecData defines an interface for general access to spectral
 datasets of differing fundamental data types and is the main data
 model used in SPLAT.}
\entityintro{SpecDataFactory}{l2}{This class creates and clones instances of SpecData that are
 appropriate to the data format of the given spectrum specification.}
\clearpage
\subsection{Classes}
\startsection{Class}{MEMSpecDataImpl}{l0}

\fbox{\parbox{\textwidth}{
\texttt{public
 class MEMSpecDataImpl extends uk.ac.starlink.splat.data.SpecDataImpl}
}} % end fbox



%\vspace{.09in}


{This class provides an implementation of SpecDataImpl to access
 spectra stored in existing memory. All values are copied into
 arrays stored in memory.
 
 The main use of this class is for temporary, generated and copied
 spectra.}
\constructors
\method{public \textbf{MEMSpecDataImpl}(\texttt{java.lang.String} \textbf{name})\label{l3}\label{l4}}
\begin{desc}Constructor - just take a symbolic name for the spectrum, no
 other significance.
\begin{itemize}
\item{Parameter
  \begin{description}
   \item[\textbf{name}]{a symbolic name for the spectrum.}
  \end{description}}
\end{itemize}
\end{desc}

\method{public \textbf{MEMSpecDataImpl}(\texttt{java.lang.String} \textbf{name}, \texttt{uk.ac.starlink.splat.data.SpecData} \textbf{spectrum})\label{l5}\label{l6}}
\begin{desc}Constructor, clone from another spectrum.
\begin{itemize}
\item{Parameters
  \begin{description}
   \item[\textbf{name}]{a symbolic name for the spectrum.}
   \item[\textbf{spectrum}]{a SpecData object to copy.}
  \end{description}}
\end{itemize}
\end{desc}

\methods
\method{public long \textbf{getAst}()\label{l7}\label{l8}}
\begin{desc}Return reference to AST frameset that defines the coordinate
 relations used by this spectrum.
\begin{itemize}
\item{Returns reference to a raw AST frameset. }
\end{itemize}
\end{desc}

\method{public double \textbf{getData}()\label{l9}\label{l10}}
\begin{desc}Return a copy of the spectrum data values.
\begin{itemize}
\item{Returns reference to the spectrum data values. }
\end{itemize}
\end{desc}

\method{public double \textbf{getDataErrors}()\label{l11}\label{l12}}
\begin{desc}Return a copy of the spectrum data errors.
\begin{itemize}
\item{Returns reference to the spectrum data values. }
\end{itemize}
\end{desc}

\method{public String \textbf{getDataFormat}()\label{l13}\label{l14}}
\begin{desc}Return the data format.
\begin{itemize}
\item{Returns the String "MEMORY". }
\end{itemize}
\end{desc}

\method{public int \textbf{getDims}()\label{l15}\label{l16}}
\begin{desc}Return the data array dimensionality (always length of
 spectrum).
\begin{itemize}
\item{Returns integer array of size 1 returning the number of data
                 values available. }
\end{itemize}
\end{desc}

\method{public String \textbf{getFullName}()\label{l17}\label{l18}}
\begin{desc}Return the full name of spectrum. For memory spectra this has
 no real meaning (i.e. no disk file or URL) so always returns a
 string reminding users that they need to save it.
\begin{itemize}
\item{Returns the String "Memory spectrum". }
\end{itemize}
\end{desc}

\method{public String \textbf{getShortName}()\label{l19}\label{l20}}
\begin{desc}Return a symbolic name.
\begin{itemize}
\item{Returns a symbolic name for the spectrum. Based on the filename. }
\end{itemize}
\end{desc}

\method{public void \textbf{save}()\label{l21}\label{l22}}
\begin{desc}Save is just a copy for this class.
\begin{itemize}
\item{{Exceptions}
  \begin{itemize}
   \item{\vspace{-.6ex}\texttt{uk.ac.starlink.splat.util.SplatException} - never thrown for this implementation.}
  \end{itemize}
}
\end{itemize}
\end{desc}

\method{public void \textbf{setData}(\texttt{double[]} \textbf{data}, \texttt{double[]} \textbf{coords})\label{l23}\label{l24}}
\begin{desc}Set the spectrum data. No errors.
\begin{itemize}
\item{Parameters
  \begin{description}
   \item[\textbf{data}]{the spectrum data values.}
   \item[\textbf{coords}]{the spectrum coordinates, one per data value.}
  \end{description}}
\end{itemize}
\end{desc}

\method{public void \textbf{setData}(\texttt{double[]} \textbf{data}, \texttt{double[]} \textbf{coords}, \texttt{double[]} \textbf{errors})\label{l25}\label{l26}}
\begin{desc}Set the spectrum data. With errors.
\begin{itemize}
\item{Parameters
  \begin{description}
   \item[\textbf{data}]{the spectrum data values.}
   \item[\textbf{coords}]{the spectrum coordinates, one per data value.}
   \item[\textbf{errors}]{the errors of the spectrum data values.}
  \end{description}}
\end{itemize}
\end{desc}

\startsection{Class}{SpecData}{l1}

\fbox{\parbox{\textwidth}{
\texttt{public
 class SpecData implements SpecDataAccess, AnalyticSpectrum, java.io.Serializable}
}} % end fbox



\vspace{.09in}


{SpecData defines an interface for general access to spectral
 datasets of differing fundamental data types and is the main data
 model used in SPLAT.
 
 It uses a derived class of SpecDataImpl to a supported data format
 (i.e. FITS, NDF and text files) to give generalised access to:
 \begin{itemize}

   \item{ the spectrum data
   }
\item{ the associated data errors
   }
\item{ the coordinate of any data point
   }
\item{ the spectrum properties (i.e. related values)
 }
\end{itemize}

 
 It should always be used when dealing with spectral data to avoid
 any specialised knowledge of the data format.
 
 Missing data, or gaps in the spectrum, are indicated using the
 special value SpecData.BAD. Generally useful code should always
 test for this in the data values (otherwise you'll see numeric
 problems as BAD is the lowest possible double value).
 
 Matching of data values between this spectrum and anothers
 coordinates can currently be done using the evalYData and
 evalYDataArray method of the AnalyticSpectrum interface (but note
 that at present this uses a simple interpolation scheme, so
 shouldn't be used for analysis, except when the interpolated
 spectrum is an analytic one, such as a polynomial).
 
 Each object records a series of properties that define how the
 spectrum should be rendered (i.e. line colour, thickness, style,
 plotting style, whether to show an errors as bars etc.). These are
 stored in any serialized versions of this class. Rendering using
 the Grf object primitives is performed by this class for spectra
 and error bars.
 
 Facilities to store the association between this spectrum and the
 various plots that it is currently associated with are also provided
 (but see the SpecList or GlobalSpecPlotList classes for ways to
 structure the control of many spectra, or many spectra and many
 plots).
 
 General utilities for converting coordinates and looking up values
 are provided, as are methods for specialised functions like
 formatting and unformatting values. This allows you to avoid
 understanding what is returned as a value from a user
 interaction as formatting and unformatting match the units of the
 spectral axes (which can therefore be in esoteric units, like RA
 or Dec).}
\fields{Fields}
\field{public static final double \textbf{BAD}}{Value of BAD (missing) data.}
\field{public static final int \textbf{LINE\_THICKNESS}}{Set or query line thickness.}
\field{public static final int \textbf{LINE\_STYLE}}{Set or query line drawing style.}
\field{public static final int \textbf{LINE\_COLOUR}}{Set or query line colour.}
\field{public static final int \textbf{PLOT\_STYLE}}{Set or query line drawing style.}
\field{public static final int \textbf{LINE\_ALPHA\_BLEND}}{Set or query alpha blending fraction.}
\field{public static final int \textbf{ERROR\_COLOUR}}{Set or query error bar colour.}
\field{public static final int \textbf{POLYLINE}}{Use polyline plotting style.}
\field{public static final int \textbf{HISTOGRAM}}{Use histogram plotting style.}
\field{public static final int \textbf{UNCLASSIFIED}}{Spectrum is unclassified.}
\field{public static final int \textbf{TARGET}}{Spectrum is a target (observation).}
\field{public static final int \textbf{ARC}}{Spectrum is an arc.}
\field{public static final int \textbf{SKY}}{Spectrum is a twilight sky exposure.}
\field{public static final int \textbf{POLYNOMIAL}}{Spectrum is a polynomial.}
\field{public static final int \textbf{LINEFIT}}{Spectrum is a line fit.}
\field{public static final int \textbf{USERTYPE}}{Spectrum is a user defined type.}

\constructors
\method{public \textbf{SpecData}(\texttt{uk.ac.starlink.splat.data.SpecDataImpl} \textbf{impl})\label{l27}\label{l28}}
\begin{desc}Create an instance using the data in a given SpecDataImpl
 object.
\begin{itemize}
\item{Parameter
  \begin{description}
   \item[\textbf{impl}]{an concrete implementation of a SpecDataImpl
             class that is accessing spectral data in of 
             some format.}
  \end{description}}
\end{itemize}
\end{desc}

\methods
\method{public void \textbf{addPlot}(\texttt{uk.ac.starlink.splat.plot.Plot} \textbf{plot})\label{l29}\label{l30}}
\begin{desc}Add a Plot reference to the list of known views of this
 spectrum.
\begin{itemize}
\item{Parameter
  \begin{description}
   \item[\textbf{plot}]{reference to a Plot}
  \end{description}}
\end{itemize}
\end{desc}

\method{public int \textbf{bound}(\texttt{double} \textbf{xcoord})\label{l31}\label{l32}}
\begin{desc}Locate the indices of the two coordinates that lie closest to a
 given coordinate. In the case of an exact match then both
 indices are returned as the same value.
\begin{itemize}
\item{Parameter
  \begin{description}
   \item[\textbf{xcoord}]{the coordinate value to bound.}
  \end{description}}
\end{itemize}
\begin{itemize}
\item{Returns array of two integers, the lower and upper indices. }
\end{itemize}
\end{desc}

\method{public void \textbf{drawSpec}(\texttt{uk.ac.starlink.splat.plot.Grf} \textbf{grf}, \texttt{long} \textbf{plot}, \texttt{double[]} \textbf{limits})\label{l33}\label{l34}}
\begin{desc}Draw the spectrum onto the given widget using a suitable AST
 GRF object.
\begin{itemize}
\item{Parameters
  \begin{description}
   \item[\textbf{grf}]{Grf object that can be drawn into using AST
            primitives.}
   \item[\textbf{plot}]{reference to AstPlot defining transformation from
             physical coordinates into graphics coordinates.}
   \item[\textbf{limits}]{limits of the region to draw in physical
               coordinates (e.g. user defined ranges), used to
               clip graphics.}
  \end{description}}
\end{itemize}
\end{desc}

\method{public double \textbf{evalYData}(\texttt{double} \textbf{x})\label{l35}\label{l36}}
\begin{desc}Return the value of the spectrum at an arbitrary X position.
\begin{itemize}
\item{Parameter
  \begin{description}
   \item[\textbf{x}]{the coordiante at which to evaluate this spectrum.}
  \end{description}}
\end{itemize}
\begin{itemize}
\item{Returns data value of this spectrum at the given coordinate. }
\end{itemize}
\end{desc}

\method{public double \textbf{evalYDataArray}(\texttt{double[]} \textbf{x})\label{l37}\label{l38}}
\begin{desc}Return the value of the spectrum evaluated at series of
 arbitrary X positions.
\begin{itemize}
\item{Parameter
  \begin{description}
   \item[\textbf{x}]{the coordiantes at which to evaluate this spectrum.}
  \end{description}}
\end{itemize}
\begin{itemize}
\item{Returns data values of this spectrum at the given coordinates. }
\end{itemize}
\end{desc}

\method{public String \textbf{format}(\texttt{int} \textbf{axis}, \texttt{long} \textbf{plot}, \texttt{double} \textbf{value})\label{l39}\label{l40}}
\begin{desc}Convert a coordinate value into a formatted String suitable for
 a given axis (could be celestial coordinates for example).
\begin{itemize}
\item{Parameters
  \begin{description}
   \item[\textbf{axis}]{the axis to use for formatting rules.}
   \item[\textbf{value}]{the value.}
   \item[\textbf{plot}]{AST plot that defines the coordinate formats.}
  \end{description}}
\end{itemize}
\begin{itemize}
\item{Returns the formatted value. }
\end{itemize}
\end{desc}

\method{public String \textbf{formatInterpolatedLookup}(\texttt{int} \textbf{xg}, \texttt{long} \textbf{plot})\label{l41}\label{l42}}
\begin{desc}Return interpolated physical values (i.e. wavelength and data
 value) that correspond to a graphics X coordinate.
\begin{itemize}
\item{Parameters
  \begin{description}
   \item[\textbf{xg}]{X graphics coordinate}
   \item[\textbf{plot}]{AST plot needed to transform graphics position
            into physical coordinates}
  \end{description}}
\end{itemize}
\begin{itemize}
\item{Returns the physical coordinate and value. These are
 linearly interpolated. }
\end{itemize}
\end{desc}

\method{public String \textbf{formatLookup}(\texttt{int} \textbf{xg}, \texttt{long} \textbf{plot})\label{l43}\label{l44}}
\begin{desc}Lookup the physical values (i.e. wavelength and data value)
 that correspond to a graphics X coordinate. Value is returned
 as formatted strings for the selected axis (could be sky
 coordinates for instance).
\begin{itemize}
\item{Parameters
  \begin{description}
   \item[\textbf{xg}]{X graphics coordinate}
   \item[\textbf{plot}]{AST plot needed to transform graphics position
             into physical coordinates}
  \end{description}}
\end{itemize}
\begin{itemize}
\item{Returns array of two Strings, the formatted wavelength and
         data values }
\end{itemize}
\end{desc}

\method{public double \textbf{getAlphaBlend}()\label{l45}\label{l46}}
\begin{desc}Get the alpha blending fraction.
\begin{itemize}
\item{Returns the current alpha blending fraction. }
\end{itemize}
\end{desc}

\method{public ASTJ \textbf{getAst}()\label{l47}\label{l48}}
\begin{desc}Get reference to ASTJ object set up to specify coordinate
 transformations.
\begin{itemize}
\item{Returns ASTJ object describing axis transformations. }
\end{itemize}
\end{desc}

\method{public String \textbf{getDataFormat}()\label{l49}\label{l50}}
\begin{desc}Return the data format as a suitable name.
\begin{itemize}
\item{Returns the data format as a simple string (FITS, NDF, etc.). }
\end{itemize}
\end{desc}

\method{public double \textbf{getErrorColour}()\label{l51}\label{l52}}
\begin{desc}Get the colour used when drawing error bars.
\begin{itemize}
\item{Returns the error bar colour (an RGB integer). }
\end{itemize}
\end{desc}

\method{public String \textbf{getFullName}()\label{l53}\label{l54}}
\begin{desc}Get the full name for spectrum (cannot edit, this is usually
 the filename).
\begin{itemize}
\item{Returns the full name. }
\end{itemize}
\end{desc}

\method{public double \textbf{getFullRange}()\label{l55}\label{l56}}
\begin{desc}Get the data ranges in the X and Y axes with standard
 deviation.
\begin{itemize}
\item{Returns reference to array of 4 values, xlow, xhigh, ylow, yhigh. }
\end{itemize}
\end{desc}

\method{public double \textbf{getLineColour}()\label{l57}\label{l58}}
\begin{desc}Get the colour of the line to be used when plotting the
 spectrum.
\begin{itemize}
\item{Returns the line colour (an RGB integer). }
\end{itemize}
\end{desc}

\method{public double \textbf{getLineStyle}()\label{l59}\label{l60}}
\begin{desc}Get the line type to be used when plotting the spectrum.
 The meaning of this value is not defined.
\begin{itemize}
\item{Returns the line style in use. }
\end{itemize}
\end{desc}

\method{public double \textbf{getLineThickness}()\label{l61}\label{l62}}
\begin{desc}Get the width of the line to be used when plotting the
 spectrum. The meaning of this value isn't defined.
\begin{itemize}
\item{Returns the line thickness. }
\end{itemize}
\end{desc}

\method{public int \textbf{getPlotStyle}()\label{l63}\label{l64}}
\begin{desc}Get the value of plotStyle.
\begin{itemize}
\item{Returns the current plotting type (SpecData.POLYLINE or
            SpecData.HISTOGRAM). }
\end{itemize}
\end{desc}

\method{public double \textbf{getRange}()\label{l65}\label{l66}}
\begin{desc}Get the data ranges in the X and Y axes. Does not include data
 errors.
\begin{itemize}
\item{Returns reference to array of 4 values, xlow, xhigh, ylow, yhigh. }
\end{itemize}
\end{desc}

\method{public SpecData \textbf{getSect}(\texttt{java.lang.String} \textbf{name}, \texttt{double[]} \textbf{ranges})\label{l67}\label{l68}}
\begin{desc}Create a new spectrum by extracting sections of this spectrum.
 The section extents are defined in physical coordinates.
 Each section will not contain any values that lie outside of
 its physical coordinate range.
 
 The spectrum created here is not added to any lists or created
 with any configuration other than the default values (i.e. you
 must do this part yourself) and is only keep in memory.
\begin{itemize}
\item{Parameters
  \begin{description}
   \item[\textbf{name}]{short name for the spectrum.}
   \item[\textbf{ranges}]{an array of pairs of physical coordinates. These
               define the extents of the ranges to extract.}
  \end{description}}
\end{itemize}
\begin{itemize}
\item{Returns a spectrum that contains data only from the given
         ranges. }
\end{itemize}
\end{desc}

\method{public String \textbf{getShortName}()\label{l69}\label{l70}}
\begin{desc}Get a symbolic name for spectrum.
\begin{itemize}
\item{Returns the short name. }
\end{itemize}
\end{desc}

\method{public SpecDataImpl \textbf{getSpecDataImpl}()\label{l71}\label{l72}}
\begin{desc}Return the SpecDataImpl object so that it can expose very data
 specific methods (if needed, you really shouldn't use this).
\begin{itemize}
\item{Returns SpecDataImpl object defining the data access used by
         this instance. }
\end{itemize}
\end{desc}

\method{public int \textbf{getType}()\label{l73}\label{l74}}
\begin{desc}Get the type of the spectral data.
\begin{itemize}
\item{Returns type either UNCLASSIFIED, TARGET, ARC, SKY, POLYNOMIAL or
                     LINEFIT or USERTYPE. }
\end{itemize}
\end{desc}

\method{public double \textbf{getXData}()\label{l75}\label{l76}}
\begin{desc}Get references to spectrum X data (i.e. the coordinates).
\begin{itemize}
\item{Returns reference to spectrum X data. }
\end{itemize}
\end{desc}

\method{public double \textbf{getYData}()\label{l77}\label{l78}}
\begin{desc}Get references to spectrum Y data (i.e. the data values).
\begin{itemize}
\item{Returns reference to spectrum Y data. }
\end{itemize}
\end{desc}

\method{public double \textbf{getYDataErrors}()\label{l79}\label{l80}}
\begin{desc}Get references to spectrum Y data errors (i.e. the data
 errors).
\begin{itemize}
\item{Returns reference to spectrum Y data errors. }
\end{itemize}
\end{desc}

\method{public boolean \textbf{haveYDataErrors}()\label{l81}\label{l82}}
\begin{desc}Return if data errors are available.
\begin{itemize}
\item{Returns true if Y data has errors. }
\end{itemize}
\end{desc}

\method{public boolean \textbf{isDrawErrorBars}()\label{l83}\label{l84}}
\begin{desc}Find out if we're drawing error bars.
\begin{itemize}
\item{Returns true if we're drawing error bars. }
\end{itemize}
\end{desc}

\method{public boolean \textbf{isUseInAutoRanging}()\label{l85}\label{l86}}
\begin{desc}Find out if we should be used when determining an auto-range.
\begin{itemize}
\item{Returns whether this spectrum is used when auto-ranging. }
\end{itemize}
\end{desc}

\method{public double \textbf{lookup}(\texttt{int} \textbf{xg}, \texttt{long} \textbf{plot})\label{l87}\label{l88}}
\begin{desc}Lookup the physical values (i.e. wavelength and data value)
 that correspond to a graphics X coordinate.
\begin{itemize}
\item{Parameters
  \begin{description}
   \item[\textbf{xg}]{X graphics coordinate}
   \item[\textbf{plot}]{AST plot needed to transform graphics position
             into physical coordinates}
  \end{description}}
\end{itemize}
\begin{itemize}
\item{Returns array of two doubles. The wavelength and data values. }
\end{itemize}
\end{desc}

\method{public int \textbf{plotCount}()\label{l89}\label{l90}}
\begin{desc}Get the number of plots currently using this spectrum.
\begin{itemize}
\item{Returns number of plot currently using this spectrum. }
\end{itemize}
\end{desc}

\method{public void \textbf{removePlot}(\texttt{int} \textbf{index})\label{l91}\label{l92}}
\begin{desc}Remove a Plot reference.
\begin{itemize}
\item{Parameter
  \begin{description}
   \item[\textbf{index}]{of the Plot}
  \end{description}}
\end{itemize}
\end{desc}

\method{public void \textbf{removePlot}(\texttt{uk.ac.starlink.splat.plot.Plot} \textbf{plot})\label{l93}\label{l94}}
\begin{desc}Remove a Plot reference.
\begin{itemize}
\item{Parameter
  \begin{description}
   \item[\textbf{plot}]{reference to a Plot}
  \end{description}}
\end{itemize}
\end{desc}

\method{public void \textbf{save}()\label{l95}\label{l96}}
\begin{desc}Save the spectrum to disk (if supported). Uses the current
 state of the spectrum (i.e. any file names etc.) to decide how
 to do this.
\begin{itemize}
\item{{Exceptions}
  \begin{itemize}
   \item{\vspace{-.6ex}\texttt{uk.ac.starlink.splat.util.SplatException} - thown if an error occurs during save.}
  \end{itemize}
}
\end{itemize}
\end{desc}

\method{public void \textbf{setAlphaBlend}(\texttt{double} \textbf{alphaBlend})\label{l97}\label{l98}}
\begin{desc}Set the alpha blending fraction of the line used plot the spectrum.
\begin{itemize}
\item{Parameter
  \begin{description}
   \item[\textbf{alphaBlend}]{the alpha belanding fraction (0.0 to 1.0).}
  \end{description}}
\end{itemize}
\end{desc}

\method{public void \textbf{setDrawErrorBars}(\texttt{boolean} \textbf{state})\label{l99}\label{l100}}
\begin{desc}Set whether the spectrum should have its errors drawn as error
 bars, or not. If no errors are available then this is always
 false.
\begin{itemize}
\item{Parameter
  \begin{description}
   \item[\textbf{state}]{true to draw error bars, if possible.}
  \end{description}}
\end{itemize}
\end{desc}

\method{public void \textbf{setErrorColour}(\texttt{int} \textbf{errorColour})\label{l101}\label{l102}}
\begin{desc}Set the colour index to be used when drawing error bars.
\begin{itemize}
\item{Parameter
  \begin{description}
   \item[\textbf{errorColour}]{the colour as an RGB integer.}
  \end{description}}
\end{itemize}
\end{desc}

\method{public void \textbf{setKnownNumberProperty}(\texttt{int} \textbf{what}, \texttt{java.lang.Number} \textbf{value})\label{l103}\label{l104}}
\begin{desc}Set a known numeric spectral property.
\begin{itemize}
\item{Parameters
  \begin{description}
   \item[\textbf{what}]{either LINE\_THICKNESS, LINE\_STYLE, LINE\_COLOUR,
             PLOT\_STYLE, LINE\_ALPHA\_BLEND or ERROR\_COLOUR.}
   \item[\textbf{value}]{container for numeric value. These depend on
              property being set.}
  \end{description}}
\end{itemize}
\end{desc}

\method{public void \textbf{setLineColour}(\texttt{int} \textbf{lineColour})\label{l105}\label{l106}}
\begin{desc}Set the colour index to be used when plotting the spectrum.
\begin{itemize}
\item{Parameter
  \begin{description}
   \item[\textbf{lineColour}]{the colour as an RGB integer.}
  \end{description}}
\end{itemize}
\end{desc}

\method{public void \textbf{setLineStyle}(\texttt{double} \textbf{lineStyle})\label{l107}\label{l108}}
\begin{desc}Set the style of the line used to plot the spectrum.
\begin{itemize}
\item{Parameter
  \begin{description}
   \item[\textbf{lineStyle}]{the line style to be used when plotting.}
  \end{description}}
\end{itemize}
\end{desc}

\method{public void \textbf{setLineThickness}(\texttt{double} \textbf{lineThickness})\label{l109}\label{l110}}
\begin{desc}Set the thickness of the line used to plot the spectrum.
\begin{itemize}
\item{Parameter
  \begin{description}
   \item[\textbf{lineThickness}]{the line width to be used when plotting.}
  \end{description}}
\end{itemize}
\end{desc}

\method{public void \textbf{setPlotStyle}(\texttt{int} \textbf{style})\label{l111}\label{l112}}
\begin{desc}Set the type of spectral lines that are drawn (these can be
 polylines or histogram-like, simple markers are a possibility
 for a future implementation). The value should be one of the
 symbolic constants "POLYLINE" and "HISTOGRAM".
\begin{itemize}
\item{Parameter
  \begin{description}
   \item[\textbf{style}]{one of the symbolic contants SpecData.POLYLINE or
             SpecData.HISTOGRAM.}
  \end{description}}
\end{itemize}
\end{desc}

\method{public void \textbf{setShortName}(\texttt{java.lang.String} \textbf{shortName})\label{l113}\label{l114}}
\begin{desc}Change the symbolic name of a spectrum.
\begin{itemize}
\item{Parameter
  \begin{description}
   \item[\textbf{shortName}]{new short name for the spectrum.}
  \end{description}}
\end{itemize}
\end{desc}

\method{public void \textbf{setType}(\texttt{int} \textbf{type})\label{l115}\label{l116}}
\begin{desc}Set the type of the spectral data.
\begin{itemize}
\item{Parameter
  \begin{description}
   \item[\textbf{type}]{either UNCLASSIFIED, TARGET, ARC, SKY, POLYNOMIAL or
                    LINEFIT or USERTYPE.}
  \end{description}}
\end{itemize}
\end{desc}

\method{public void \textbf{setUseInAutoRanging}(\texttt{boolean} \textbf{state})\label{l117}\label{l118}}
\begin{desc}Set whether the spectrum should be used when deriving
 auto-ranged values.
\begin{itemize}
\item{Parameter
  \begin{description}
   \item[\textbf{state}]{if true then this spectrum should be used when
             determining data limits (artificial spectra could
             have unreasonable ranges when extrapolated).}
  \end{description}}
\end{itemize}
\end{desc}

\method{public int \textbf{size}()\label{l119}\label{l120}}
\begin{desc}Get the size of the spectrum.
\begin{itemize}
\item{Returns number of coordinate positions in spectrum. }
\end{itemize}
\end{desc}

\method{public double \textbf{unFormat}(\texttt{int} \textbf{axis}, \texttt{long} \textbf{plot}, \texttt{java.lang.String} \textbf{value})\label{l121}\label{l122}}
\begin{desc}Convert a formatted coordinate string into a double precision
 value (could be celestial coordinates for example).
\begin{itemize}
\item{Parameters
  \begin{description}
   \item[\textbf{axis}]{the axis to use for formatting rules.}
   \item[\textbf{plot}]{AST plot that defines the coordinate formats.}
   \item[\textbf{value}]{the formatted value.}
  \end{description}}
\end{itemize}
\begin{itemize}
\item{Returns the unformatted value. }
\end{itemize}
\end{desc}

\startsection{Class}{SpecDataFactory}{l2}

\fbox{\parbox{\textwidth}{
\texttt{public
 class SpecDataFactory}
}} % end fbox



\vspace{.09in}


{This class creates and clones instances of SpecData that are
 appropriate to the data format of the given spectrum specification.
 
 The specification is usual the file name plus extension and any
 qualifiers (such as HDS object path, or FITS extension
 number). This is parsed and identified by the InputNameParser
 class.}
\methods
\method{public SpecData \textbf{get}(\texttt{java.lang.String} \textbf{specspec})\label{l123}\label{l124}}
\begin{desc}Check the format of the incoming specification and create an
  instance of SpecData for it.
\begin{itemize}
\item{Parameter
  \begin{description}
   \item[\textbf{specspec}]{the specification of the spectrum to be
                  opened (i.e. file.fits, file.sdf,
                  file.fits[2], file.more.ext\_1 etc.).}
  \end{description}}
\end{itemize}
\begin{itemize}
\item{Returns the SpecData object created from the given
          specification. }
\item{{Exceptions}
  \begin{itemize}
   \item{\vspace{-.6ex}\texttt{uk.ac.starlink.splat.util.SplatException} - thrown if specification does not
             specify a spectrum.}
  \end{itemize}
}
\end{itemize}
\end{desc}

\method{public SpecData \textbf{getClone}(\texttt{uk.ac.starlink.splat.data.SpecData} \textbf{source}, \texttt{java.lang.String} \textbf{specspec})\label{l125}\label{l126}}
\begin{desc}Create an clone of an existing spectrum by transforming it into
 another implementation format. The destination format is
 decided using the usual rules on specification string.
\begin{itemize}
\item{Parameters
  \begin{description}
   \item[\textbf{source}]{SpecData object to be cloned.}
   \item[\textbf{specspec}]{name of the resultant clone (defines
                 implementation type).}
  \end{description}}
\end{itemize}
\begin{itemize}
\item{Returns the cloned SpecData object. }
\item{{Exceptions}
  \begin{itemize}
   \item{\vspace{-.6ex}\texttt{uk.ac.starlink.splat.util.SplatException} - maybe thrown if there are problems
            creating the new SpecData object or the implementation.}
  \end{itemize}
}
\end{itemize}
\end{desc}

\method{public static SpecDataFactory \textbf{getReference}()\label{l127}\label{l128}}
\begin{desc}Return reference to the only allowed instance of this class.
\begin{itemize}
\item{Returns reference to only instance of this class. }
\end{itemize}
\end{desc}

\clearpage


\section{Package uk.ac.starlink.splat.plot}

\vspace{.13in}
\hbox{\textbf{Classes}}
\vspace{.13in}
\entityintro{PlotControl}{l129}{A PlotControl object consists of a Plot inside a scrolled pane and
 various controls in a panel.}
\clearpage
\subsection{Classes}
\startsection{Class}{PlotControl}{l129}

\fbox{\parbox{\textwidth}{
\texttt{public
 class PlotControl extends javax.swing.JPanel implements MouseMotionTracker, uk.ac.starlink.splat.iface.SpecListener, FigureListener, PlotScaledListener, java.awt.event.ActionListener}
}} % end fbox



\vspace{.09in}


{A PlotControl object consists of a Plot inside a scrolled pane and
 various controls in a panel. The Plot allows you to display many
 spectra. Plots wrapped by this object are given a name that is
 unique within an application (\textless plotn\textgreater ) and are assigned an
 identifying number (0 upwards).
 
 The controls in the panel allow you to:
 \begin{itemize}
 \item{ apply independent scales in X and Y, thus zooming and
        scrolling the Plot,\textless $/$li\textgreater 
 }
   \item{  get a continuous readout of the cursor position,
   }
\item{  display a vertical hair,\textless $/$li\textgreater 
   }
\item{  select the current spectrum and see a list of those
        displayed,\textless $/$li\textgreater 
   }
\item{  apply percentile cuts to the data limits.\textless $/$li\textgreater 
 }
\end{itemize}

 Methods are provided for scaling about the current centre, zooming
 in to a given region as well as just setting the X and Y scale
 factors.
 
 You can also scroll to centre on a given X coordinate (or the one
 shown in the X coordinate readout field) and make the Plot fit
 itself to the current viewable height and$/$or width.
 
 A not-very exhaustive set of methods for querying the contents of
 the Plot (e.g. getting access to the plotted spectra) and adding
 and removing spectra from the Plot are provided.
 
 Finally this object can make the Plot produce a postscript and JPEG
 output copy of itself.
 
 See the Plot object for a description of the facilities it
 provides.}

\constructors
\method{public \textbf{PlotControl}()\label{l130}\label{l131}}
\begin{desc}Create a Plot, adding spectra later.
\end{desc}

\method{public \textbf{PlotControl}(\texttt{uk.ac.starlink.splat.data.SpecDataComp} \textbf{spectra})\label{l132}\label{l133}}
\begin{desc}Plot a list of spectra referenced in a SpecDataComp object.
\begin{itemize}
\item{Parameter
  \begin{description}
   \item[\textbf{spectra}]{reference to SpecDataComp object that is
                wrapping the spectra to display.}
  \end{description}}
\end{itemize}
\end{desc}

\method{public \textbf{PlotControl}(\texttt{java.lang.String} \textbf{file})\label{l134}\label{l135}}
\begin{desc}Plot a spectrum using its filename specification. Standalone
 constructor.
\begin{itemize}
\item{Parameter
  \begin{description}
   \item[\textbf{file}]{name of file containing spectrum.}
  \end{description}}
\end{itemize}
\end{desc}

\methods
\method{public void \textbf{actionPerformed}(\texttt{java.awt.event.ActionEvent} \textbf{e})\label{l136}\label{l137}}
\begin{desc}Respond to selection of a new spectrum as the current one. This
 makes the selected spectrum current in the global list.
\begin{itemize}
\item{Parameter
  \begin{description}
   \item[\textbf{e}]{object describing the event.}
  \end{description}}
\end{itemize}
\end{desc}

\method{public void \textbf{addSpectrum}(\texttt{uk.ac.starlink.splat.data.SpecData} \textbf{spec})\label{l138}\label{l139}}
\begin{desc}Add a SpecData reference to the list of displayed spectra.
\begin{itemize}
\item{Parameter
  \begin{description}
   \item[\textbf{spec}]{reference to a spectrum.}
  \end{description}}
\end{itemize}
\end{desc}

\method{public void \textbf{centreOnXCoordinate}()\label{l140}\label{l141}}
\begin{desc}Centre on the value shown in the xValue field, if possible.
\end{desc}

\method{public void \textbf{centreOnXCoordinate}(\texttt{java.lang.String} \textbf{coord})\label{l142}\label{l143}}
\begin{desc}Centre on a given coordinate. The coordinate is a formatted
 value.
\begin{itemize}
\item{Parameter
  \begin{description}
   \item[\textbf{coord}]{coordinate to centre on.}
  \end{description}}
\end{itemize}
\end{desc}

\method{public void \textbf{figureChanged}(\texttt{uk.ac.starlink.splat.plot.FigureChangedEvent} \textbf{e})\label{l144}\label{l145}}
\begin{desc}Send when the zoom figure is changed. Do nothing.
\begin{itemize}
\item{Parameter
  \begin{description}
   \item[\textbf{e}]{object describing the event.}
  \end{description}}
\end{itemize}
\end{desc}

\method{public void \textbf{figureCreated}(\texttt{uk.ac.starlink.splat.plot.FigureChangedEvent} \textbf{e})\label{l146}\label{l147}}
\begin{desc}Sent when the zoom figure is created. Do nothing.
\begin{itemize}
\item{Parameter
  \begin{description}
   \item[\textbf{e}]{object describing the event.}
  \end{description}}
\end{itemize}
\end{desc}

\method{public void \textbf{figureRemoved}(\texttt{uk.ac.starlink.splat.plot.FigureChangedEvent} \textbf{e})\label{l148}\label{l149}}
\begin{desc}Sent when a zoom interaction is complete. Events by mouse
 button 3 are assumed to mean unzoom.
\begin{itemize}
\item{Parameter
  \begin{description}
   \item[\textbf{e}]{object describing the event.}
  \end{description}}
\end{itemize}
\end{desc}

\method{public void \textbf{finalize}()\label{l150}\label{l151}}
\begin{desc}Release any locally allocated resources and references.
\begin{itemize}
\item{{Exceptions}
  \begin{itemize}
   \item{\vspace{-.6ex}\texttt{java.lang.Throwable} - if finalize fails.}
  \end{itemize}
}
\end{itemize}
\end{desc}

\method{public void \textbf{fitToHeight}()\label{l152}\label{l153}}
\begin{desc}Fit spectrum to the displayed height.
\end{desc}

\method{public void \textbf{fitToWidth}()\label{l154}\label{l155}}
\begin{desc}Fit spectrum to the displayed width.
\end{desc}

\method{public void \textbf{fitToWidthAndHeight}()\label{l156}\label{l157}}
\begin{desc}Fit spectrum to the displayed width and height at same time.
\end{desc}

\method{public double \textbf{getCentre}()\label{l158}\label{l159}}
\begin{desc}Get the centre of the current view.
\begin{itemize}
\item{Returns array of two doubles, the X and Y coordinates of the
 centre of the current view (view coordinates). }
\end{itemize}
\end{desc}

\method{public PlotConfig \textbf{getConfig}()\label{l160}\label{l161}}
\begin{desc}Get the Graphics configuration object.
\begin{itemize}
\item{Returns reference to the current graphics configuration object. }
\end{itemize}
\end{desc}

\method{public SpecData \textbf{getCurrentSpectrum}()\label{l162}\label{l163}}
\begin{desc}Return the current spectrum (top of the combobox of names).
 This method is for use in picking a spectrum from all those
 displayed by any associated toolboxes that can only work with
 one spectrum.
\begin{itemize}
\item{Returns the current spectrum. }
\end{itemize}
\end{desc}

\method{public double \textbf{getDisplayCoordinates}()\label{l164}\label{l165}}
\begin{desc}Get the physical coordinate limits of the complete Plot
 (i.e. the whole plot, including zoomed regions that you cannot
 see).
\begin{itemize}
\item{Returns array of four doubles, the lower X, lower Y, upper X
         and upper Y coordinates. }
\end{itemize}
\end{desc}

\method{public int \textbf{getIdentifier}()\label{l166}\label{l167}}
\begin{desc}Return the integer identifier of this plot (unique among plots).
\begin{itemize}
\item{Returns integer identifier. }
\end{itemize}
\end{desc}

\method{public String \textbf{getName}()\label{l168}\label{l169}}
\begin{desc}Return the name of this plot (unique among plots).
\begin{itemize}
\item{Returns name of the Plot. }
\end{itemize}
\end{desc}

\method{public Plot \textbf{getPlot}()\label{l170}\label{l171}}
\begin{desc}Return the Plot reference.
\begin{itemize}
\item{Returns reference to the Plot. }
\end{itemize}
\end{desc}

\method{public SpecDataComp \textbf{getSpecDataComp}()\label{l172}\label{l173}}
\begin{desc}Return reference to SpecDataComp.
\begin{itemize}
\item{Returns the SpecDataComp object used to wrapp all spectra
         displayed in the Plot. }
\end{itemize}
\end{desc}

\method{public double \textbf{getViewCoordinates}()\label{l174}\label{l175}}
\begin{desc}Get the physical coordinates limits of the current view of the
 Plot (i.e. what you can see).
\begin{itemize}
\item{Returns array of four doubles, the lower X, lower Y, upper X
         and upper Y coordinates. }
\end{itemize}
\end{desc}

\method{public JViewport \textbf{getViewport}()\label{l176}\label{l177}}
\begin{desc}Get reference to the JViewport being used to display the
 spectra.
\begin{itemize}
\item{Returns reference to the JViewport. }
\end{itemize}
\end{desc}

\method{public double \textbf{getViewRange}()\label{l178}\label{l179}}
\begin{desc}Get the coordinate range displayed in the current view.
\begin{itemize}
\item{Returns array of two doubles, the lower and upper limits of the
        Plot in X coordinates. }
\end{itemize}
\end{desc}

\method{public float \textbf{getXScale}()\label{l180}\label{l181}}
\begin{desc}Get the current X scale factor.
\begin{itemize}
\item{Returns the current X scale factor. }
\end{itemize}
\end{desc}

\method{public float \textbf{getYScale}()\label{l182}\label{l183}}
\begin{desc}Get the current Y scale factor.
\begin{itemize}
\item{Returns the current Y scale factor. }
\end{itemize}
\end{desc}

\method{public boolean \textbf{isDisplayed}(\texttt{uk.ac.starlink.splat.data.SpecData} \textbf{spec})\label{l184}\label{l185}}
\begin{desc}Say if a spectrum is being plotted.
\begin{itemize}
\item{Parameter
  \begin{description}
   \item[\textbf{spec}]{the spectrum tio check.}
  \end{description}}
\end{itemize}
\begin{itemize}
\item{Returns true if the SpecData object is being displayed. }
\end{itemize}
\end{desc}

\method{public void \textbf{maybeScaleAboutCentre}()\label{l186}\label{l187}}
\begin{desc}Set the display scale factor to that shown, keeping the current
 centre of field, unless a centre operation is already pending.
\end{desc}

\method{public void \textbf{plotScaleChanged}(\texttt{uk.ac.starlink.splat.plot.PlotScaleChangedEvent} \textbf{e})\label{l188}\label{l189}}
\begin{desc}Make any adjustments needed to respond to a change in scale by
 the Plot.
\begin{itemize}
\item{Parameter
  \begin{description}
   \item[\textbf{e}]{object describing event.}
  \end{description}}
\end{itemize}
\end{desc}

\method{public void \textbf{print}()\label{l190}\label{l191}}
\begin{desc}Make a printable copy of the Plot content.
\end{desc}

\method{public void \textbf{removeSpectrum}(\texttt{int} \textbf{index})\label{l192}\label{l193}}
\begin{desc}Remove a spectrum from the Plot.
\begin{itemize}
\item{Parameter
  \begin{description}
   \item[\textbf{index}]{of the spectrum.}
  \end{description}}
\end{itemize}
\end{desc}

\method{public void \textbf{removeSpectrum}(\texttt{uk.ac.starlink.splat.data.SpecData} \textbf{spec})\label{l194}\label{l195}}
\begin{desc}Remove a spectrum from the Plot.
\begin{itemize}
\item{Parameter
  \begin{description}
   \item[\textbf{spec}]{reference to a spectrum.}
  \end{description}}
\end{itemize}
\end{desc}

\method{public void \textbf{resetScales}()\label{l196}\label{l197}}
\begin{desc}Reset the apparent scales of the Plot. This means that
 whatever the current size of the Plot its scale becomes 1x1.
\end{desc}

\method{public void \textbf{scaleScale}(\texttt{float} \textbf{xs}, \texttt{float} \textbf{ys})\label{l198}\label{l199}}
\begin{desc}Scale the scale factors.
\begin{itemize}
\item{Parameters
  \begin{description}
   \item[\textbf{xs}]{value to scale the X scale factor by.}
   \item[\textbf{ys}]{value to scale the Y scale factor by.}
  \end{description}}
\end{itemize}
\end{desc}

\method{public void \textbf{setCentre}(\texttt{double} \textbf{x}, \texttt{double} \textbf{y})\label{l200}\label{l201}}
\begin{desc}Set the viewport to show a given position as the centre.
\begin{itemize}
\item{Parameters
  \begin{description}
   \item[\textbf{x}]{the X coordinate of new centre (view coordinates).}
   \item[\textbf{y}]{the Y coordinate of new centre (view coordinates).}
  \end{description}}
\end{itemize}
\end{desc}

\method{public void \textbf{setConfig}(\texttt{uk.ac.starlink.splat.iface.PlotConfig} \textbf{config})\label{l202}\label{l203}}
\begin{desc}Set the Graphics configuration object.
\begin{itemize}
\item{Parameter
  \begin{description}
   \item[\textbf{config}]{the graphics configuration object.}
  \end{description}}
\end{itemize}
\end{desc}

\method{public void \textbf{setScale}()\label{l204}\label{l205}}
\begin{desc}Set the display scale factor to that shown.
\end{desc}

\method{public void \textbf{setScale}(\texttt{float} \textbf{xs}, \texttt{float} \textbf{ys})\label{l206}\label{l207}}
\begin{desc}Set the display scale factors. Also scales the plot.
\begin{itemize}
\item{Parameters
  \begin{description}
   \item[\textbf{xs}]{X scale factor.}
   \item[\textbf{ys}]{Y scale factor.}
  \end{description}}
\end{itemize}
\end{desc}

\method{public void \textbf{setSpecDataComp}(\texttt{uk.ac.starlink.splat.data.SpecDataComp} \textbf{spectra})\label{l208}\label{l209}}
\begin{desc}Set the SpecDataComp component used.
\begin{itemize}
\item{Parameter
  \begin{description}
   \item[\textbf{spectra}]{new SpecDataComp object to use for wrapping any
                spectra displayed in the Plot.}
  \end{description}}
\end{itemize}
\begin{itemize}
\item{{Exceptions}
  \begin{itemize}
   \item{\vspace{-.6ex}\texttt{uk.ac.starlink.splat.util.SplatException} - thrown if problems reading spectral
            data.}
  \end{itemize}
}
\end{itemize}
\end{desc}

\method{public void \textbf{setXScale}(\texttt{float} \textbf{xs})\label{l210}\label{l211}}
\begin{desc}Set the X scale factor.
\begin{itemize}
\item{Parameter
  \begin{description}
   \item[\textbf{xs}]{the new X scale factor.}
  \end{description}}
\end{itemize}
\end{desc}

\method{public void \textbf{setYScale}(\texttt{float} \textbf{ys})\label{l212}\label{l213}}
\begin{desc}Set the Y scale factor.
\begin{itemize}
\item{Parameter
  \begin{description}
   \item[\textbf{ys}]{the new Y scale factor.}
  \end{description}}
\end{itemize}
\end{desc}

\method{public int \textbf{specCount}()\label{l214}\label{l215}}
\begin{desc}Get the number of spectrum currently plotted.
\begin{itemize}
\item{Returns number of spectra being displayed. }
\end{itemize}
\end{desc}

\method{public void \textbf{spectrumAdded}(\texttt{uk.ac.starlink.splat.iface.SpecChangedEvent} \textbf{e})\label{l216}\label{l217}}
\begin{desc}A new spectrum is added. Do nothing, until it is added to this
 plot.
\begin{itemize}
\item{Parameter
  \begin{description}
   \item[\textbf{e}]{object describing the event.}
  \end{description}}
\end{itemize}
\end{desc}

\method{public void \textbf{spectrumChanged}(\texttt{uk.ac.starlink.splat.iface.SpecChangedEvent} \textbf{e})\label{l218}\label{l219}}
\begin{desc}React to a spectrum property change, if one of ours.
\begin{itemize}
\item{Parameter
  \begin{description}
   \item[\textbf{e}]{object describing the event.}
  \end{description}}
\end{itemize}
\end{desc}

\method{public void \textbf{spectrumCurrent}(\texttt{uk.ac.starlink.splat.iface.SpecChangedEvent} \textbf{e})\label{l220}\label{l221}}
\begin{desc}React to a change in the global current spectrum. Do nothing.
\begin{itemize}
\item{Parameter
  \begin{description}
   \item[\textbf{e}]{object describing the event.}
  \end{description}}
\end{itemize}
\end{desc}

\method{public void \textbf{spectrumRemoved}(\texttt{uk.ac.starlink.splat.iface.SpecChangedEvent} \textbf{e})\label{l222}\label{l223}}
\begin{desc}React to a spectrum being removed, if one of ours.
\begin{itemize}
\item{Parameter
  \begin{description}
   \item[\textbf{e}]{object describing the event.}
  \end{description}}
\end{itemize}
\end{desc}

\method{public String \textbf{toString}()\label{l224}\label{l225}}
\begin{desc}Override toString to return the name of this plot (unique
 among plots).
\begin{itemize}
\item{Returns name of this plot. }
\end{itemize}
\end{desc}

\method{public void \textbf{updateCoords}(\texttt{java.lang.String} \textbf{x}, \texttt{java.lang.String} \textbf{y})\label{l226}\label{l227}}
\begin{desc}Update the displayed coordinates (implementation from PlotMouseMotion).
\begin{itemize}
\item{Parameters
  \begin{description}
   \item[\textbf{x}]{the X coordinate value to show.}
   \item[\textbf{y}]{the Y coordinate value to show.}
  \end{description}}
\end{itemize}
\end{desc}

\method{public void \textbf{updatePlot}()\label{l228}\label{l229}}
\begin{desc}Update the plot. Should be called when events that require the
 Plot to redraw itself occur (i.e. when spectra are added or
 removed and when the Plot configuration is changed).
\end{desc}

\method{public void \textbf{zoomAbout}(\texttt{int} \textbf{xIncrement}, \texttt{int} \textbf{yIncrement}, \texttt{double} \textbf{x}, \texttt{double} \textbf{y})\label{l230}\label{l231}}
\begin{desc}Increment scales in both dimensions about a centre.
\begin{itemize}
\item{Parameters
  \begin{description}
   \item[\textbf{xIncrement}]{increment for X scale factor.}
   \item[\textbf{yIncrement}]{increment for Y scale factor.}
   \item[\textbf{x}]{X coordinate to zoom about.}
   \item[\textbf{y}]{Y coordinate to zoom about.}
  \end{description}}
\end{itemize}
\end{desc}

\method{public void \textbf{zoomAboutTheCentre}(\texttt{int} \textbf{xIncrement}, \texttt{int} \textbf{yIncrement})\label{l232}\label{l233}}
\begin{desc}Zoom plot about the current center by the given increments in
 the X and Y scale factors.
\begin{itemize}
\item{Parameters
  \begin{description}
   \item[\textbf{xIncrement}]{value to add to current X scale factor.}
   \item[\textbf{yIncrement}]{value to add to current Y scale factor.}
  \end{description}}
\end{itemize}
\end{desc}

\method{public void \textbf{zoomToRectangularRegion}(\texttt{java.awt.geom.Rectangle2D} \textbf{region})\label{l234}\label{l235}}
\begin{desc}Zoom$/$scroll the Plot to display a given rectangular region.
 
 If the region has zero width or height then the zoom is merely
 incremented by 1 in both dimensions. Also if the scale factors
 are increased by more than a factor of 20, then they are
 clipped to this value (the most likely action here is a user
 accidently returning a very small region).
\begin{itemize}
\item{Parameter
  \begin{description}
   \item[\textbf{region}]{the region to zoom into.}
  \end{description}}
\end{itemize}
\end{desc}

\clearpage


\section{Package uk.ac.starlink.splat.util}

\vspace{.13in}
\hbox{\textbf{Classes}}
\vspace{.13in}
\entityintro{RemoteUtilities}{l236}{This class provides utility methods for remote control based
 access to SPLAT.}
\entityintro{SpectralFileFilter}{l237}{A convenience implementation of java.io.FileFilter that filters out
 all files except for those type extensions that it knows about.}
\clearpage
\subsection{Classes}
\startsection{Class}{RemoteUtilities}{l236}

\fbox{\parbox{\textwidth}{
\texttt{public
 class RemoteUtilities}
}} % end fbox



\vspace{.09in}


{This class provides utility methods for remote control based
 access to SPLAT.}
\methods
\method{public static boolean \textbf{isListening}(\texttt{java.lang.Object[]} \textbf{contactDetails})\label{l238}\label{l239}}
\begin{desc}Find out if SPLAT is listening on the given hostname and port
 and recognises the given cookie.
\begin{itemize}
\item{Parameter
  \begin{description}
   \item[\textbf{contactDetails}]{return from readContactFile.}
  \end{description}}
\end{itemize}
\begin{itemize}
\item{Returns true if an instance of SPLAT was successfully contacted. }
\end{itemize}
\end{desc}

\method{public static Object \textbf{readContactFile}()\label{l240}\label{l241}}
\begin{desc}Parse the contact file returning its contents as an Object array.
\begin{itemize}
\item{Returns array of three Objects. These are really the hostname
         String, an Integer with the port number and a String
         containing the validation cookie. Returns null if not
         available. }
\end{itemize}
\end{desc}

\method{public static String \textbf{sendRemoteCommand}(\texttt{java.lang.Object[]} \textbf{contactDetails}, \texttt{java.lang.String} \textbf{command})\label{l242}\label{l243}}
\begin{desc}Send a single beanshell command to a remote SPLAT beanshell
 interpreter. Needs the hostname, port and cookie from the
 contact file.
\begin{itemize}
\item{Parameters
  \begin{description}
   \item[\textbf{contactDetails}]{return from readContactFile.}
   \item[\textbf{command}]{the beanshell command to execute.}
  \end{description}}
\end{itemize}
\begin{itemize}
\item{Returns the string that returns from the command, null for an error. }
\item{{Exceptions}
  \begin{itemize}
   \item{\vspace{-.6ex}\texttt{java.lang.Exception} - exception resulting from any network problems.}
  \end{itemize}
}
\end{itemize}
\end{desc}

\startsection{Class}{SpectralFileFilter}{l237}

\fbox{\parbox{\textwidth}{
\texttt{public
 class SpectralFileFilter implements java.io.FileFilter}
}} % end fbox



\vspace{.09in}


{A convenience implementation of java.io.FileFilter that filters out
 all files except for those type extensions that it knows about.
 
 Extensions are of the type ".foo", which is typically found on
 Windows and Unix boxes, but not on Macinthosh. Case is ignored.
 
 Example - create a new filter that filters out all files
 but gif and jpg image files:
 \texttt{
\mbox{}\newline
    SpectralFileFilter filter = new SpectralFileFilter(\mbox{}\newline
                   new String$\{$"gif", "jpg"$\}$, "JPEG \& GIF Images")\mbox{}\newline
 }
}
\constructors
\method{public \textbf{SpectralFileFilter}()\label{l244}\label{l245}}
\begin{desc}Creates a file filter. If no filters are added, then all
 files are accepted.
\begin{itemize}
\item{{See also}
  \begin{itemize}
   \item{\texttt{uk.ac.starlink.splat.util.SpectralFileFilter.addExtension} {
\refdefined{l246}}
}
  \end{itemize}
}
\end{itemize}
\end{desc}

\method{public \textbf{SpectralFileFilter}(\texttt{java.lang.String} \textbf{extension})\label{l247}\label{l248}}
\begin{desc}Creates a file filter that accepts files with the given extension.
 Example: new SpectralFileFilter("jpg");
\begin{itemize}
\item{Parameter
  \begin{description}
   \item[\textbf{extension}]{file extension.}
  \end{description}}
\end{itemize}
\begin{itemize}
\item{{See also}
  \begin{itemize}
   \item{\texttt{uk.ac.starlink.splat.util.SpectralFileFilter.addExtension} {
\refdefined{l246}}
}
  \end{itemize}
}
\end{itemize}
\end{desc}

\method{public \textbf{SpectralFileFilter}(\texttt{java.lang.String[]} \textbf{filters})\label{l249}\label{l250}}
\begin{desc}Creates a file filter from the given string array.
 Example: new SpectralFileFilter( String $\{$"gif", "jpg"$\}$ );
 
 Note that the "." before the extension is not needed and will
 be ignored.
\begin{itemize}
\item{Parameter
  \begin{description}
   \item[\textbf{filters}]{array of file extensions to use a filters.}
  \end{description}}
\end{itemize}
\begin{itemize}
\item{{See also}
  \begin{itemize}
   \item{\texttt{uk.ac.starlink.splat.util.SpectralFileFilter.addExtension} {
\refdefined{l246}}
}
  \end{itemize}
}
\end{itemize}
\end{desc}

\methods
\method{public boolean \textbf{accept}(\texttt{java.io.File} \textbf{pathname})\label{l251}\label{l252}}
\begin{desc}Tests whether or not the specified abstract pathname should be
 included in a pathname list.
\begin{itemize}
\item{Parameter
  \begin{description}
   \item[\textbf{pathname}]{The abstract pathname to be tested}
  \end{description}}
\end{itemize}
\begin{itemize}
\item{Returns true if and only if pathname should be included. }
\end{itemize}
\end{desc}

\method{public void \textbf{addExtension}(\texttt{java.lang.String} \textbf{extension})\label{l253}\label{l254}}
\begin{desc}Adds a filetype "dot" extension to filter against.

 For example: the following code will create a filter that filters
 out all files except those that end in ".jpg" and ".tif":
 \texttt{
\mbox{}\newline
   SpectralFileFilter filter = new SpectralFileFilter();\mbox{}\newline
   filter.addExtension("jpg");\mbox{}\newline
   filter.addExtension("tif");\mbox{}\newline
 }

 Note that the "." before the extension is not needed and will
 be ignored.
\begin{itemize}
\item{Parameter
  \begin{description}
   \item[\textbf{extension}]{a file extension to add.}
  \end{description}}
\end{itemize}
\end{desc}

\method{public String \textbf{getExtension}(\texttt{java.io.File} \textbf{f})\label{l255}\label{l256}}
\begin{desc}Return the extension portion of the file's name.
\begin{itemize}
\item{Parameter
  \begin{description}
   \item[\textbf{f}]{a File object.}
  \end{description}}
\end{itemize}
\begin{itemize}
\item{Returns the file extension. }
\item{{See also}
  \begin{itemize}
   \item{\texttt{uk.ac.starlink.splat.util.SpectralFileFilter.getExtension} {
\refdefined{l257}}
}
   \item{\texttt{java.io.FileFilter.accept} {
\refdefined{l258}}
}
  \end{itemize}
}
\end{itemize}
\end{desc}

\clearpage


\section{Package uk.ac.starlink.splat.iface}

\vspace{.13in}
\hbox{\textbf{Classes}}
\vspace{.13in}
\entityintro{GlobalSpecPlotList}{l259}{GlobalSpecPlotList is an aggregate singleton class that provides access
 to the SpecList and PlotControlList objects.}
\entityintro{SplatBrowser}{l260}{This is the main class for the SPLAT program.}
\clearpage
\subsection{Classes}
\startsection{Class}{GlobalSpecPlotList}{l259}

\fbox{\parbox{\textwidth}{
\texttt{public
 class GlobalSpecPlotList}
}} % end fbox



\vspace{.09in}


{GlobalSpecPlotList is an aggregate singleton class that provides access
 to the SpecList and PlotControlList objects. It provides integrated
 control interfaces to both these objects and provides listeners for
 objects (i.e. views) that want to be updated about changes in the
 lists of spectra or plots.}
\methods
\method{public int \textbf{add}(\texttt{int} \textbf{index}, \texttt{uk.ac.starlink.splat.data.SpecData} \textbf{spectrum})\label{l261}\label{l262}}
\begin{desc}Replace or add a spectrum. Informs any listeners of change.
\begin{itemize}
\item{Parameters
  \begin{description}
   \item[\textbf{index}]{index of the spectrum to replace. Appended to end
               if index not used.}
   \item[\textbf{spectrum}]{reference to a SpecData object.}
  \end{description}}
\end{itemize}
\begin{itemize}
\item{Returns index of the spectrum in global list, or -1. }
\end{itemize}
\end{desc}

\method{public int \textbf{add}(\texttt{uk.ac.starlink.splat.plot.PlotControl} \textbf{plot})\label{l263}\label{l264}}
\begin{desc}Add a plot (immediately after creation).
\begin{itemize}
\item{Parameter
  \begin{description}
   \item[\textbf{plot}]{reference to a PlotControl object.}
  \end{description}}
\end{itemize}
\begin{itemize}
\item{Returns global index of the plot. }
\end{itemize}
\end{desc}

\method{public int \textbf{add}(\texttt{uk.ac.starlink.splat.data.SpecData} \textbf{spectrum})\label{l265}\label{l266}}
\begin{desc}Add a spectrum to the global list. Informs any listeners.
\begin{itemize}
\item{Parameter
  \begin{description}
   \item[\textbf{spectrum}]{reference to a SpecData object.}
  \end{description}}
\end{itemize}
\begin{itemize}
\item{Returns index of the spectrum in global list. }
\end{itemize}
\end{desc}

\method{public void \textbf{addPlotListener}(\texttt{uk.ac.starlink.splat.iface.PlotListener} \textbf{l})\label{l267}\label{l268}}
\begin{desc}Registers a listener for to be informed when plots are
  created, changed or removed.
\begin{itemize}
\item{Parameter
  \begin{description}
   \item[\textbf{l}]{the PlotListener}
  \end{description}}
\end{itemize}
\end{desc}

\method{public void \textbf{addSpecListener}(\texttt{uk.ac.starlink.splat.iface.SpecListener} \textbf{l})\label{l269}\label{l270}}
\begin{desc}Registers a listener for to be informed when spectra are added
  or removed from the global list.
\begin{itemize}
\item{Parameter
  \begin{description}
   \item[\textbf{l}]{the SpecListener}
  \end{description}}
\end{itemize}
\end{desc}

\method{public void \textbf{addSpectrum}(\texttt{int} \textbf{plotIndex}, \texttt{int} \textbf{specIndex})\label{l271}\label{l272}}
\begin{desc}Add a known spectrum to a plot.
\begin{itemize}
\item{Parameters
  \begin{description}
   \item[\textbf{plotIndex}]{global index of the plot to add the spectrum to.}
   \item[\textbf{specIndex}]{global index of the spectrum.}
  \end{description}}
\end{itemize}
\end{desc}

\method{public void \textbf{addSpectrum}(\texttt{int} \textbf{plotIndex}, \texttt{uk.ac.starlink.splat.data.SpecData} \textbf{spectrum})\label{l273}\label{l274}}
\begin{desc}Add a known spectrum to a plot.
\begin{itemize}
\item{Parameters
  \begin{description}
   \item[\textbf{plotIndex}]{index of the plot to add spectrum to.}
   \item[\textbf{spectrum}]{the spectrum to add.}
  \end{description}}
\end{itemize}
\end{desc}

\method{public void \textbf{addSpectrum}(\texttt{uk.ac.starlink.splat.plot.PlotControl} \textbf{plot}, \texttt{int} \textbf{specIndex})\label{l275}\label{l276}}
\begin{desc}Add a known spectrum to a plot.
\begin{itemize}
\item{Parameters
  \begin{description}
   \item[\textbf{plot}]{the plot to add the spectrum to.}
   \item[\textbf{specIndex}]{global index of the spectrum.}
  \end{description}}
\end{itemize}
\end{desc}

\method{public void \textbf{addSpectrum}(\texttt{uk.ac.starlink.splat.plot.PlotControl} \textbf{plot}, \texttt{uk.ac.starlink.splat.data.SpecData} \textbf{spectrum})\label{l277}\label{l278}}
\begin{desc}Add a known spectrum to a plot.
\begin{itemize}
\item{Parameters
  \begin{description}
   \item[\textbf{plot}]{the plot to add spectrum to.}
   \item[\textbf{spectrum}]{the spectrum to add.}
  \end{description}}
\end{itemize}
\end{desc}

\method{public String \textbf{getFullName}(\texttt{int} \textbf{index})\label{l279}\label{l280}}
\begin{desc}Return the full (usually diskfile) name of a spectrum.
\begin{itemize}
\item{Parameter
  \begin{description}
   \item[\textbf{index}]{index of the spectrum.}
  \end{description}}
\end{itemize}
\begin{itemize}
\item{Returns full name of the spectrum. }
\end{itemize}
\end{desc}

\method{public PlotControl \textbf{getPlot}(\texttt{int} \textbf{index})\label{l281}\label{l282}}
\begin{desc}Get a plot reference by index.
\begin{itemize}
\item{Parameter
  \begin{description}
   \item[\textbf{index}]{index of required plot.}
  \end{description}}
\end{itemize}
\begin{itemize}
\item{Returns reference to PlotControl object. }
\end{itemize}
\end{desc}

\method{public int \textbf{getPlotIndex}(\texttt{int} \textbf{identifier})\label{l283}\label{l284}}
\begin{desc}See if a plot with the given identifier exists, and if so
 return its index. Note identifiers are a unique integer among
 all plots and are not the plot index (which varies as plots are
 removed).
\begin{itemize}
\item{Parameter
  \begin{description}
   \item[\textbf{identifier}]{the identifier of the plot whose existence is
                   to be checked.}
  \end{description}}
\end{itemize}
\begin{itemize}
\item{Returns the plot index, or -1 if not located. }
\end{itemize}
\end{desc}

\method{public int \textbf{getPlotIndex}(\texttt{uk.ac.starlink.splat.plot.PlotControl} \textbf{plot})\label{l285}\label{l286}}
\begin{desc}Get an index for a plot.
\begin{itemize}
\item{Parameter
  \begin{description}
   \item[\textbf{plot}]{PlotControl object to look up.}
  \end{description}}
\end{itemize}
\begin{itemize}
\item{Returns global index of object, if found. }
\end{itemize}
\end{desc}

\method{public String \textbf{getPlotName}(\texttt{int} \textbf{index})\label{l287}\label{l288}}
\begin{desc}Return the name of a plot.
\begin{itemize}
\item{Parameter
  \begin{description}
   \item[\textbf{index}]{index of spectrum.}
  \end{description}}
\end{itemize}
\begin{itemize}
\item{Returns name of the plot. }
\end{itemize}
\end{desc}

\method{public static GlobalSpecPlotList \textbf{getReference}()\label{l289}\label{l290}}
\begin{desc}Return reference to the only allowed instance of this class.
\begin{itemize}
\item{Returns reference to the only instance of this class. }
\end{itemize}
\end{desc}

\method{public String \textbf{getShortName}(\texttt{int} \textbf{index})\label{l291}\label{l292}}
\begin{desc}Return the symbolic name of a spectrum.
\begin{itemize}
\item{Parameter
  \begin{description}
   \item[\textbf{index}]{index of th spectrum.}
  \end{description}}
\end{itemize}
\begin{itemize}
\item{Returns short name of te spectrum. }
\end{itemize}
\end{desc}

\method{public SpecData \textbf{getSpectrum}(\texttt{int} \textbf{index})\label{l293}\label{l294}}
\begin{desc}Get a spectrum reference by index.
\begin{itemize}
\item{Parameter
  \begin{description}
   \item[\textbf{index}]{index of spectrum to retrieve.}
  \end{description}}
\end{itemize}
\begin{itemize}
\item{Returns the SpecData object or null. }
\end{itemize}
\end{desc}

\method{public int \textbf{getSpectrumIndex}(\texttt{uk.ac.starlink.splat.data.SpecData} \textbf{spectrum})\label{l295}\label{l296}}
\begin{desc}Get an index for a spectrum.
\begin{itemize}
\item{Parameter
  \begin{description}
   \item[\textbf{spectrum}]{SpecData object whose index is needed.}
  \end{description}}
\end{itemize}
\begin{itemize}
\item{Returns index of the spectrum or -1. }
\end{itemize}
\end{desc}

\method{public int \textbf{getSpectrumIndex}(\texttt{java.lang.String} \textbf{shortName})\label{l297}\label{l298}}
\begin{desc}Get an index for a spectrum using its short name.
\begin{itemize}
\item{Parameter
  \begin{description}
   \item[\textbf{shortName}]{spectrum short name.}
  \end{description}}
\end{itemize}
\begin{itemize}
\item{Returns index of spectrum if found, -1 otherwise. }
\end{itemize}
\end{desc}

\method{public boolean \textbf{isDisplaying}(\texttt{int} \textbf{plotIndex}, \texttt{int} \textbf{specIndex})\label{l299}\label{l300}}
\begin{desc}Return if a plot is displaying a given spectrum.
\begin{itemize}
\item{Parameters
  \begin{description}
   \item[\textbf{plotIndex}]{global index of the plot to check.}
   \item[\textbf{specIndex}]{global index of the spectrum.}
  \end{description}}
\end{itemize}
\begin{itemize}
\item{Returns true of the plot is displaying the spectrum. }
\end{itemize}
\end{desc}

\method{public boolean \textbf{isDisplaying}(\texttt{int} \textbf{plotIndex}, \texttt{uk.ac.starlink.splat.data.SpecData} \textbf{spectrum})\label{l301}\label{l302}}
\begin{desc}Return if a plot is displaying a given spectrum.
\begin{itemize}
\item{Parameters
  \begin{description}
   \item[\textbf{plotIndex}]{global index of the plot to check.}
   \item[\textbf{spectrum}]{the spectrum to check.}
  \end{description}}
\end{itemize}
\begin{itemize}
\item{Returns true if the plot is displaying the spectrum. }
\end{itemize}
\end{desc}

\method{public int \textbf{plotCount}()\label{l303}\label{l304}}
\begin{desc}Return the number of plots in the global list.
\begin{itemize}
\item{Returns number of current plots. }
\end{itemize}
\end{desc}

\method{public int \textbf{remove}(\texttt{uk.ac.starlink.splat.plot.PlotControl} \textbf{plot})\label{l305}\label{l306}}
\begin{desc}Remove a plot (immediately after destruction).
\begin{itemize}
\item{Parameter
  \begin{description}
   \item[\textbf{plot}]{reference to the plot that has been deleted.}
  \end{description}}
\end{itemize}
\begin{itemize}
\item{Returns index of the removed plot. }
\end{itemize}
\end{desc}

\method{public void \textbf{removePlotListener}(\texttt{uk.ac.starlink.splat.iface.PlotListener} \textbf{l})\label{l307}\label{l308}}
\begin{desc}Remove a listener for plot changes.
\begin{itemize}
\item{Parameter
  \begin{description}
   \item[\textbf{l}]{the PlotListener}
  \end{description}}
\end{itemize}
\end{desc}

\method{public void \textbf{removeSpecListener}(\texttt{uk.ac.starlink.splat.iface.SpecListener} \textbf{l})\label{l309}\label{l310}}
\begin{desc}Remove a listener for changes in the global list of spectra.
\begin{itemize}
\item{Parameter
  \begin{description}
   \item[\textbf{l}]{the SpecListener}
  \end{description}}
\end{itemize}
\end{desc}

\method{public int \textbf{removeSpectrum}(\texttt{int} \textbf{index})\label{l311}\label{l312}}
\begin{desc}Remove a spectrum. Note listeners are informed immediately
  before the spectrum is removed (so the lists remain current).
\begin{itemize}
\item{Parameter
  \begin{description}
   \item[\textbf{index}]{index of the spectrum to remove.}
  \end{description}}
\end{itemize}
\begin{itemize}
\item{Returns index of the spectrum that was removed. }
\end{itemize}
\end{desc}

\method{public void \textbf{removeSpectrum}(\texttt{int} \textbf{plotIndex}, \texttt{int} \textbf{specIndex})\label{l313}\label{l314}}
\begin{desc}Remove a known spectrum from a plot.
\begin{itemize}
\item{Parameters
  \begin{description}
   \item[\textbf{plotIndex}]{global index of the plot to remove the
                   spectrum from.}
   \item[\textbf{specIndex}]{global index of the spectrum.}
  \end{description}}
\end{itemize}
\end{desc}

\method{public void \textbf{removeSpectrum}(\texttt{int} \textbf{plotIndex}, \texttt{uk.ac.starlink.splat.data.SpecData} \textbf{spectrum})\label{l315}\label{l316}}
\begin{desc}Remove a known spectrum from a plot.
\begin{itemize}
\item{Parameters
  \begin{description}
   \item[\textbf{plotIndex}]{global index of the plot.}
   \item[\textbf{spectrum}]{the spectrum to remove.}
  \end{description}}
\end{itemize}
\end{desc}

\method{public void \textbf{removeSpectrum}(\texttt{uk.ac.starlink.splat.plot.PlotControl} \textbf{plot}, \texttt{int} \textbf{specIndex})\label{l317}\label{l318}}
\begin{desc}Remove a known spectrum from a plot.
\begin{itemize}
\item{Parameters
  \begin{description}
   \item[\textbf{plot}]{the plot to remove the spectrum from.}
   \item[\textbf{specIndex}]{global index of the spectrum.}
  \end{description}}
\end{itemize}
\end{desc}

\method{public void \textbf{removeSpectrum}(\texttt{uk.ac.starlink.splat.plot.PlotControl} \textbf{plot}, \texttt{uk.ac.starlink.splat.data.SpecData} \textbf{spectrum})\label{l319}\label{l320}}
\begin{desc}Remove a known spectrum from a plot.
\begin{itemize}
\item{Parameters
  \begin{description}
   \item[\textbf{plot}]{the plot to remove the spectrum from.}
   \item[\textbf{spectrum}]{the spectrum to remove.}
  \end{description}}
\end{itemize}
\end{desc}

\method{public int \textbf{removeSpectrum}(\texttt{uk.ac.starlink.splat.data.SpecData} \textbf{spectrum})\label{l321}\label{l322}}
\begin{desc}Remove a spectrum. Note listeners are informed immediately
  before the spectrum is removed (so the lists remain current).
\begin{itemize}
\item{Parameter
  \begin{description}
   \item[\textbf{spectrum}]{reference to the spectrum to remove.}
  \end{description}}
\end{itemize}
\begin{itemize}
\item{Returns index of the spectrum that was removed. }
\end{itemize}
\end{desc}

\method{public void \textbf{setCurrentSpectrum}(\texttt{int} \textbf{index})\label{l323}\label{l324}}
\begin{desc}Set the index of the current spectrum. Changes to this are
 sent to any SpecListeners.
\begin{itemize}
\item{Parameter
  \begin{description}
   \item[\textbf{index}]{index of the current spectrum.}
  \end{description}}
\end{itemize}
\end{desc}

\method{public void \textbf{setDrawErrorBars}(\texttt{uk.ac.starlink.splat.data.SpecData} \textbf{spectrum}, \texttt{boolean} \textbf{show})\label{l325}\label{l326}}
\begin{desc}Set if a spectrum should be displaying it's error bars or
  not.
\begin{itemize}
\item{Parameters
  \begin{description}
   \item[\textbf{spectrum}]{the spectrum to modify.}
   \item[\textbf{show}]{whether to show errorbars.}
  \end{description}}
\end{itemize}
\end{desc}

\method{public void \textbf{setKnownNumberProperty}(\texttt{uk.ac.starlink.splat.data.SpecData} \textbf{spectrum}, \texttt{int} \textbf{what}, \texttt{java.lang.Number} \textbf{value})\label{l327}\label{l328}}
\begin{desc}Set a known Number property of a spectrum.
\begin{itemize}
\item{Parameters
  \begin{description}
   \item[\textbf{spectrum}]{SpecData object to change.}
   \item[\textbf{what}]{the property to change, see
               SpecData.setKnownNumberProperty.}
   \item[\textbf{value}]{Object containing the new value.}
  \end{description}}
\end{itemize}
\begin{itemize}
\item{{See also}
  \begin{itemize}
   \item{\texttt{uk.ac.starlink.splat.iface.GlobalSpecPlotList.SpecData.setKnownNumberProperty()} {
\refdefined{l329}}
}
  \end{itemize}
}
\end{itemize}
\end{desc}

\method{public void \textbf{setShortName}(\texttt{int} \textbf{index}, \texttt{java.lang.String} \textbf{name})\label{l330}\label{l331}}
\begin{desc}Set the symbolic name of a spectrum.
\begin{itemize}
\item{Parameters
  \begin{description}
   \item[\textbf{index}]{index of the spectrum to modify.}
   \item[\textbf{name}]{the new short name for the spectrum.}
  \end{description}}
\end{itemize}
\end{desc}

\method{public int \textbf{specCount}()\label{l332}\label{l333}}
\begin{desc}Return the number of spectra in the global list.
\begin{itemize}
\item{Returns the number of spectra in the global list. }
\end{itemize}
\end{desc}

\method{public boolean \textbf{specKnown}(\texttt{java.lang.String} \textbf{name})\label{l334}\label{l335}}
\begin{desc}See if a spectrum is already known by a specification (i.e. filename).
\begin{itemize}
\item{Parameter
  \begin{description}
   \item[\textbf{name}]{name to check.}
  \end{description}}
\end{itemize}
\begin{itemize}
\item{Returns true if the name is known. }
\end{itemize}
\end{desc}

\startsection{Class}{SplatBrowser}{l260}

\fbox{\parbox{\textwidth}{
\texttt{public
 class SplatBrowser extends javax.swing.JFrame}
}} % end fbox



\vspace{.09in}


{This is the main class for the SPLAT program. It creates the
 browser interface that displays and controls the global lists of
 the currently available spectra and plots.
 
 Using the menus and controls of this interface, spectra can be
 opened, removed and copied into the global list. Plots can created,
 have spectra added and be closed.
 
 Groups of selected spectra can have their display inspected and
 changed using the controls of the related \#SplatSelectedProperties
 object.
 
 There are also a series of global, rather than spectra, specific
 tools that are made available via a toolbar. These include an
 animator tool and tools for performing simple spectral arithmetic,
 plus more trivial options, like choosing the look and feel.
 
 The actual display and interactive analysis of spectra takes place
 in the plots (see \#PlotControlFrame, \#PlotControl and \#Plot).}
\fields{Serializable Fields}
\field{private Timer \textbf{waitTimer}}{Timer for used for event queue actions.}
\field{private int \textbf{filesDone}}{Number of spectra currently loaded by the addChosenSpectra method.}
\field{private int \textbf{plotProgress}}{}

\constructors
\method{public \textbf{SplatBrowser}()\label{l336}\label{l337}}
\begin{desc}Create a browser with no existing spectra.
\end{desc}

\method{public \textbf{SplatBrowser}(\texttt{java.lang.String[]} \textbf{inspec})\label{l338}\label{l339}}
\begin{desc}Constructor, with list of spectra to initialise. All spectra
 given this way are displayed in a single plot.
\begin{itemize}
\item{Parameter
  \begin{description}
   \item[\textbf{inspec}]{list of spectra to add. If null then none are
                added.}
  \end{description}}
\end{itemize}
\end{desc}

\methods
\method{public boolean \textbf{addSpectrum}(\texttt{java.lang.String} \textbf{name})\label{l340}\label{l341}}
\begin{desc}Add a new spectrum to the global list.
\begin{itemize}
\item{Parameter
  \begin{description}
   \item[\textbf{name}]{the name (i.e. file specification) of the spectrum
              to add.}
  \end{description}}
\end{itemize}
\begin{itemize}
\item{Returns true if spectrum is added, false otherwise. }
\end{itemize}
\end{desc}

\method{public void \textbf{animateSelectedSpectra}()\label{l342}\label{l343}}
\begin{desc}Display a tool for selecting from the global list of spectra
 and then repeatable displaying the selected sequence. This is
 meant to allow the browsing of a series of spectra, or simulate
 blink comparison sequence.
\end{desc}

\method{public void \textbf{copySelectedSpectra}()\label{l344}\label{l345}}
\begin{desc}Make copies of all the selected spectra. These are memory
 copies.
\end{desc}

\method{public void \textbf{deSelectAllPlots}()\label{l346}\label{l347}}
\begin{desc}Deselect all the plots.
\end{desc}

\method{public void \textbf{deSelectAllSpectra}()\label{l348}\label{l349}}
\begin{desc}Deselect all the spectra.
\end{desc}

\method{public void \textbf{displaySelectedSpectra}()\label{l350}\label{l351}}
\begin{desc}Display all the spectra that are selected in the global list
 view. Each spectrum is displayed in a new plot.
\end{desc}

\method{public int \textbf{displaySpectra}(\texttt{int} \textbf{id}, \texttt{java.lang.String} \textbf{list})\label{l352}\label{l353}}
\begin{desc}Display a list of spectra given by their file names, in a plot
 specified by its identifier. The file names are assumed to be
 in a space separated list stored in a single String. If the
 plot doesn't exist then it is created and the identifier of
 that plot is returned (which will be different from the one
 requested).
\begin{itemize}
\item{Parameters
  \begin{description}
   \item[\textbf{id}]{the plot identifier number.}
   \item[\textbf{list}]{the list of spectra file names (or disk
             specifications).}
  \end{description}}
\end{itemize}
\begin{itemize}
\item{Returns the id of the plot that the spectra are displayed
         in, -1 if it fails. }
\end{itemize}
\end{desc}

\method{public int \textbf{displaySpectra}(\texttt{java.lang.String} \textbf{list})\label{l354}\label{l355}}
\begin{desc}Display a list of spectra given by their file names. The file
 names are assumed to be in a space separated list stored in a
 single String.
\begin{itemize}
\item{Parameter
  \begin{description}
   \item[\textbf{list}]{the list of spectra file names (or disk
             specifications).}
  \end{description}}
\end{itemize}
\begin{itemize}
\item{Returns the identifier of the plot that the spectra are
         displayed in, -1 if it fails. }
\end{itemize}
\end{desc}

\method{public PlotControlFrame \textbf{displaySpectrum}(\texttt{uk.ac.starlink.splat.data.SpecData} \textbf{spectrum})\label{l356}\label{l357}}
\begin{desc}Display a spectrum in a new plot.
\begin{itemize}
\item{Parameter
  \begin{description}
   \item[\textbf{spectrum}]{The spectrum to display.}
  \end{description}}
\end{itemize}
\begin{itemize}
\item{Returns the plot that the spectrum is displayed in. }
\end{itemize}
\end{desc}

\method{public PlotControlFrame \textbf{displaySpectrum}(\texttt{java.lang.String} \textbf{spectrum})\label{l358}\label{l359}}
\begin{desc}Add and display a new spectrum in a new plot.
\begin{itemize}
\item{Parameter
  \begin{description}
   \item[\textbf{spectrum}]{the name (i.e. file specification) of the spectrum
                 to add and display.}
  \end{description}}
\end{itemize}
\begin{itemize}
\item{Returns the plot that the spectrum is displayed in. }
\end{itemize}
\end{desc}

\method{public int \textbf{getSelectedPlots}()\label{l360}\label{l361}}
\begin{desc}Get a list of all plots current selected in the plot table.
\begin{itemize}
\item{Returns list of indices of selected plot, otherwise null. }
\end{itemize}
\end{desc}

\method{public int \textbf{getSelectedSpectra}()\label{l362}\label{l363}}
\begin{desc}Get a list of all spectra current selected in the global list.
\begin{itemize}
\item{Returns list of indices of selected spectra, otherwise null. }
\end{itemize}
\end{desc}

\method{public void \textbf{multiDisplaySelectedSpectra}(\texttt{boolean} \textbf{fit})\label{l364}\label{l365}}
\begin{desc}Display all the currently selected spectra in the currently
 selected plot, or create a new plot for them.
\begin{itemize}
\item{Parameter
  \begin{description}
   \item[\textbf{fit}]{whether to make all spectra fit the width and height
            of the plot}
  \end{description}}
\end{itemize}
\end{desc}

\method{public void \textbf{removePlot}(\texttt{uk.ac.starlink.splat.iface.PlotControlFrame} \textbf{plot})\label{l366}\label{l367}}
\begin{desc}Remove a plot from the global list.
\begin{itemize}
\item{Parameter
  \begin{description}
   \item[\textbf{plot}]{reference to the frame (i.e. window) containing the
             plot (these have a one to one relationship).}
  \end{description}}
\end{itemize}
\end{desc}

\method{public void \textbf{removeSelectedPlots}()\label{l368}\label{l369}}
\begin{desc}Remove the currently selected plots from the global list and
 this interface.
\end{desc}

\method{public void \textbf{removeSelectedSpectra}()\label{l370}\label{l371}}
\begin{desc}Remove the currently selected spectra from the global list and
 this interface.
\end{desc}

\method{public void \textbf{saveSpectrum}(\texttt{int} \textbf{globalIndex}, \texttt{java.lang.String} \textbf{spectrum})\label{l372}\label{l373}}
\begin{desc}Save a spectrum to disk file.
\begin{itemize}
\item{Parameters
  \begin{description}
   \item[\textbf{globalIndex}]{the index on the global list of the spectrum
                    to save.}
   \item[\textbf{spectrum}]{the file to write the spectrum into.}
  \end{description}}
\end{itemize}
\end{desc}

\method{public void \textbf{selectAllPlots}()\label{l374}\label{l375}}
\begin{desc}Select all the plots.
\end{desc}

\method{public void \textbf{selectAllSpectra}()\label{l376}\label{l377}}
\begin{desc}Select all the spectra.
\end{desc}

\method{public void \textbf{showBinaryMathsWindow}()\label{l378}\label{l379}}
\begin{desc}Show window for performing simple binary mathematical
 operations between spectra.
\end{desc}

\method{public void \textbf{showUnaryMathsWindow}()\label{l380}\label{l381}}
\begin{desc}Show window for performing simple unary mathematical
 operations on a spectrum.
\end{desc}

\end{document}
