\documentclass[twoside,11pt]{article}

% ? Specify used packages
\usepackage{graphicx}        %  Use this one for final production.
% \usepackage[draft]{graphicx} %  Use this one for drafting.
% ? End of specify used packages

% Nice fonts -- during development.
\usepackage{fancyheadings}
\usepackage[latin1]{inputenc}
\usepackage{ae}
\usepackage{cmbright}

\pagestyle{myheadings}
\raggedbottom

% -----------------------------------------------------------------------------
% ? Document identification
% Fixed part
\newcommand{\stardoccategory}  {Starlink User Note}
\newcommand{\stardocinitials}  {SUN}
\newcommand{\stardocsource}    {sun\stardocnumber}
\newcommand{\stardoccopyright}
{Copyright \copyright\ 2001-2003 Council for the Central Laboratory of the
Research Councils}

% Variable part - replace [xxx] as appropriate.
\newcommand{\stardocnumber}    {254.0}
\newcommand{\stardocauthors}   {Alasdair Allan}
\newcommand{\stardocdate}      {21 Apr 2004}
\newcommand{\stardoctitle}     {FROG - A Time Series Analysis Package}
\newcommand{\stardocversion}   {1.0}
\newcommand{\stardocmanual}    {User's Manual}
\newcommand{\stardocabstract}  {
The \textsf{FROG} application is an extensible analysis and display tool for time series, and is part of the next generation Starlink software work and released under the GNU Public License (GPL). 

Written in Java, it has been designed for the Web and Grid Service era as an extensible, pluggable, tool for time series analysis and display. With an integrated SOAP server the packages functionality is exposed to the user for use in their own code, and to be used remotely over the Grid, as part of the Virtual Observatory (VO).}
% ? End of document identification
% -----------------------------------------------------------------------------

% +
%  Name:
%     sun.tex
%
%  Purpose:
%     Template for Starlink User Note (SUN) documents.
%     Refer to SUN/199
%
%  Authors:
%     AJC: A.J.Chipperfield (Starlink, RAL)
%     BLY: M.J.Bly (Starlink, RAL)
%     PWD: Peter W. Draper (Starlink, Durham University)
%
%  History:
%     17-JAN-1996 (AJC):
%        Original with hypertext macros, based on MDL plain originals.
%     16-JUN-1997 (BLY):
%        Adapted for LaTeX2e.
%        Added picture commands.
%     13-AUG-1998 (PWD):
%        Converted for use with LaTeX2HTML version 98.2 and
%        Star2HTML version 1.3.
%      1-FEB-2000 (AJC):
%        Add Copyright statement in LaTeX
%     {Add further history here}
%
% -

\newcommand{\stardocname}{\stardocinitials /\stardocnumber}
\markboth{\stardocname}{\stardocname}
\setlength{\textwidth}{160mm}
\setlength{\textheight}{230mm}
\setlength{\topmargin}{-2mm}
\setlength{\oddsidemargin}{0mm}
\setlength{\evensidemargin}{0mm}
\setlength{\parindent}{0mm}
\setlength{\parskip}{\medskipamount}
\setlength{\unitlength}{1mm}

% -----------------------------------------------------------------------------
%  Hypertext definitions.
%  ======================
%  These are used by the LaTeX2HTML translator in conjunction with star2html.

%  Comment.sty: version 2.0, 19 June 1992
%  Selectively in/exclude pieces of text.
%
%  Author
%    Victor Eijkhout                                      <eijkhout@cs.utk.edu>
%    Department of Computer Science
%    University Tennessee at Knoxville
%    104 Ayres Hall
%    Knoxville, TN 37996
%    USA

%  Do not remove the %begin{latexonly} and %end{latexonly} lines (used by
%  LaTeX2HTML to signify text it shouldn't process).
%begin{latexonly}
\makeatletter
\def\makeinnocent#1{\catcode`#1=12 }
\def\csarg#1#2{\expandafter#1\csname#2\endcsname}

\def\ThrowAwayComment#1{\begingroup
    \def\CurrentComment{#1}%
    \let\do\makeinnocent \dospecials
    \makeinnocent\^^L% and whatever other special cases
    \endlinechar`\^^M \catcode`\^^M=12 \xComment}
{\catcode`\^^M=12 \endlinechar=-1 %
 \gdef\xComment#1^^M{\def\test{#1}
      \csarg\ifx{PlainEnd\CurrentComment Test}\test
          \let\html@next\endgroup
      \else \csarg\ifx{LaLaEnd\CurrentComment Test}\test
            \edef\html@next{\endgroup\noexpand\end{\CurrentComment}}
      \else \let\html@next\xComment
      \fi \fi \html@next}
}
\makeatother

\def\includecomment
 #1{\expandafter\def\csname#1\endcsname{}%
    \expandafter\def\csname end#1\endcsname{}}
\def\excludecomment
 #1{\expandafter\def\csname#1\endcsname{\ThrowAwayComment{#1}}%
    {\escapechar=-1\relax
     \csarg\xdef{PlainEnd#1Test}{\string\\end#1}%
     \csarg\xdef{LaLaEnd#1Test}{\string\\end\string\{#1\string\}}%
    }}

%  Define environments that ignore their contents.
\excludecomment{comment}
\excludecomment{rawhtml}
\excludecomment{htmlonly}

%  Hypertext commands etc. This is a condensed version of the html.sty
%  file supplied with LaTeX2HTML by: Nikos Drakos <nikos@cbl.leeds.ac.uk> &
%  Jelle van Zeijl <jvzeijl@isou17.estec.esa.nl>. The LaTeX2HTML documentation
%  should be consulted about all commands (and the environments defined above)
%  except \xref and \xlabel which are Starlink specific.

\newcommand{\htmladdnormallinkfoot}[2]{#1\footnote{#2}}
\newcommand{\htmladdnormallink}[2]{#1}
\newcommand{\htmladdimg}[1]{}
\newcommand{\hyperref}[4]{#2\ref{#4}#3}
\newcommand{\htmlref}[2]{#1}
\newcommand{\htmlimage}[1]{}
\newcommand{\htmladdtonavigation}[1]{}

\newenvironment{latexonly}{}{}
\newcommand{\latex}[1]{#1}
\newcommand{\html}[1]{}
\newcommand{\latexhtml}[2]{#1}
\newcommand{\HTMLcode}[2][]{}

%  Starlink cross-references and labels.
\newcommand{\xref}[3]{#1}
\newcommand{\xlabel}[1]{}

%  LaTeX2HTML symbol.
\newcommand{\latextohtml}{\LaTeX2\texttt{HTML}}

%  Define command to re-centre underscore for Latex and leave as normal
%  for HTML (severe problems with \_ in tabbing environments and \_\_
%  generally otherwise).
\renewcommand{\_}{\texttt{\symbol{95}}}

% -----------------------------------------------------------------------------
%  Debugging.
%  =========
%  Remove % on the following to debug links in the HTML version using Latex.

% \newcommand{\hotlink}[2]{\fbox{\begin{tabular}[t]{@{}c@{}}#1\\\hline{\footnotesize #2}\end{tabular}}}
% \renewcommand{\htmladdnormallinkfoot}[2]{\hotlink{#1}{#2}}
% \renewcommand{\htmladdnormallink}[2]{\hotlink{#1}{#2}}
% \renewcommand{\hyperref}[4]{\hotlink{#1}{\S\ref{#4}}}
% \renewcommand{\htmlref}[2]{\hotlink{#1}{\S\ref{#2}}}
% \renewcommand{\xref}[3]{\hotlink{#1}{#2 -- #3}}
%end{latexonly}
% -----------------------------------------------------------------------------
% ? Document specific \newcommand or \newenvironment commands.

% FROG.
\newcommand{\FROG}{\textsf{FROG}}
\newcommand{\SPLAT}{\textsf{SPLAT}}
\newcommand{\PERIOD}{\textsf{PERIOD}}

% Major graphic (like a screen shot). Needs ".gif" and ".eps" forms.
% \latexhtml{\includegraphics[width=4.5in]{#1.eps}}{\htmladdimg{#1.gif}}
\newcommand{\mainfigure}[1]
{\begin{center}
 \latexhtml{\includegraphics[scale=0.5]{sun254.fig/#1.eps}}{\htmladdimg{../sun254.fig/#1.gif}}
 \end{center}
}

\newcommand{\clippedmainfigure}[1]
{\begin{quote}
 \latexhtml{\includegraphics[scale=0.5,clip=true]{sun254.fig/#1.eps}}{\htmladdimg{../sun254.fig/#1.gif}}
 \end{quote}
}

% Inline a graphic (like an icon). Needs ".gif" and ".eps" forms.
\newcommand{\inline}[1]
        {\latexhtml{\includegraphics[scale=0.5]{sun254.fig/#1.eps}}
        {\htmladdimg[align=center]{../sun254.fig/#1.gif}}}

% UI elements.
\newcommand{\menuitem}[1]{\textbf{#1}}
\newcommand{\submenuitem}[2]{\latexhtml{\textbf{#1$\rightarrow$#2}}{\textbf{#1=>#2}}}
\newcommand{\labelitem}[1]{\textbf{#1}}

% typed text.
\newcommand{\hitext}[1]{\texttt{#1}}

% i.e.
\newcommand{\ie}{\textit{i.e.}}

% e.g..
\newcommand{\eg}{\textit{e.g.}}

% etc.
\newcommand{\etc}{\textit{etc.}}

% Heading for a paragraph section.
\newcommand{\subheading}[1]{\textbf{\large{#1}}}

% ? End of document specific commands
% -----------------------------------------------------------------------------
%  Title Page.
%  ===========
\renewcommand{\thepage}{\roman{page}}
\begin{document}
\thispagestyle{empty}

%  Latex document header.
%  ======================
\begin{latexonly}
   CCLRC / Rutherford Appleton Laboratory \hfill \textbf{\stardocname}\\
   {\large Particle Physics \& Astronomy Research Council}\\
   {\large Starlink Project\\}
   {\large \stardoccategory\ \stardocnumber}
   \begin{flushright}
   \stardocauthors\\
   \stardocdate
   \end{flushright}
   \vspace{-4mm}
   \rule{\textwidth}{0.5mm}
   \vspace{5mm}
   \begin{center}
      {\LARGE\textbf{\stardoctitle \\ [2.5ex]}}
   \end{center}
   \vspace{5mm}

% ? Add picture here if required for the LaTeX version.
\begin{center}
\includegraphics[scale=0.6]{sun254.fig/splash.eps}
\end{center}
% ? End of picture

% ? Heading for abstract if used.
   %\vspace{10mm}
   \begin{center}
      {\Large\textbf{Abstract}}
   \end{center}
% ? End of heading for abstract.
\end{latexonly}

%  HTML documentation header.
%  ==========================
\begin{htmlonly}
   \xlabel{}
   \begin{rawhtml} <H1 ALIGN=CENTER> <FONT COLOR="#000099">\end{rawhtml}
      \stardoctitle
   \begin{rawhtml} </FONT></H1> \end{rawhtml}

% ? Add picture here if required for the hypertext version.
   \begin{center}
      \htmladdimg{../sun254.fig/splash.gif}
   \end{center}
% ? End of picture

   \begin{rawhtml} <P> <I> \end{rawhtml}
   \stardoccategory\ \stardocnumber \\
   \stardocauthors \\
   \stardocdate
   \begin{rawhtml} </I> </P> <H3> \end{rawhtml}
      \htmladdnormallink{CCLRC / Rutherford Appleton Laboratory}
                        {http://www.cclrc.ac.uk} \\
      \htmladdnormallink{Particle Physics \& Astronomy Research Council}
                        {http://www.pparc.ac.uk} \\
   \begin{rawhtml} </H3> <H2> \end{rawhtml}
      \htmladdnormallink{Starlink Project}{http://www.starlink.rl.ac.uk/}
   \begin{rawhtml} </H2> \end{rawhtml}
   \htmladdnormallink{\htmladdimg{source.gif} Retrieve hardcopy}
      {http://www.starlink.rl.ac.uk/cgi-bin/hcserver?\stardocsource}\\

%  HTML document table of contents.
%  ================================
%  Add table of contents header and a navigation button to return to this
%  point in the document (this should always go before the abstract \section).
  \label{stardoccontents}
  \begin{rawhtml}
    <HR>
    <H2>Contents</H2>
  \end{rawhtml}
  \htmladdtonavigation{\htmlref{\htmladdimg{contents_motif.gif}}
        {stardoccontents}}

% ? New section for abstract if used.
  \section{\xlabel{abstract}Abstract}
% ? End of new section for abstract
\end{htmlonly}

% -----------------------------------------------------------------------------
% ? Document Abstract. (if used)
%  ==================
\begin{center}
\stardocabstract
\end{center}
% ? End of document abstract

% -----------------------------------------------------------------------------
% ? Latex Copyright Statement
%  =========================
\begin{latexonly}
\newpage
\vspace*{\fill}
\stardoccopyright
\end{latexonly}
% ? End of Latex copyright statement

% -----------------------------------------------------------------------------
% ? Latex document Table of Contents (if used).
%  ===========================================
\newpage
\begin{latexonly}
  \setlength{\parskip}{0mm}
  \tableofcontents
  \setlength{\parskip}{\medskipamount}
  \markboth{\stardocname}{\stardocname}
\end{latexonly}
% ? End of Latex document table of contents
% -----------------------------------------------------------------------------

\cleardoublepage
\renewcommand{\thepage}{\arabic{page}}
\setcounter{page}{1}

% ? Main text


%-----------------------------------------------------------------------
\section{Overview\xlabel{overview}}

\FROG\ is a graphical tool for displaying, comparing, modifying and
analysing time series stored in FITS and TEXT file. It has been designed to provise a simple user interface for astronomers wanting to do time domain astrophysics, but still provide the powerful features found in packages like \PERIOD\ .

If you have any problems with or comments to make about \FROG\, then send
these to:
\begin{quote}
\begin{verbatim}
   frog@starlink.ac.uk
\end{verbatim}
\end{quote}

%-----------------------------------------------------------------------
\section{Getting started\xlabel{getting_started}}

The following sections describes how to display, and carry out basic analysis of, a time series. If you're new to \FROG\ then do take the time to read this through and try out the example commands.

\newpage
\subsection{Displaying a time series\xlabel{display_a_time_series}}

To start \FROG\ you should just need to type the command:
\begin{quote}
\begin{verbatim}
  % frog &
\end{verbatim}
\end{quote}
This assumes that you have \FROG\ installed on your system as part of a
standard Starlink installation. If not then you'll need follow any
installation and pre-startup instructions that you have before using
this command.

When \FROG\ appears it should look something like:

\mainfigure{main_window}

This is the main \hitext{browser} window.

To open your first time series select the \menuitem{Open Time Series} item in the \menuitem{File} menu. This will create an  open file dialog window. Just navigate to your spectrum, select it, and press \menuitem{Open} to proceed.

Alternatively you could have supplied the time series you want to use on
the command-line:
\begin{quote}
\begin{verbatim}
  % frog <time series> &
\end{verbatim}
\end{quote}
Which will read the time series stored in the named file. \FROG\ will read
series stored in a FITS or a TEXT files.

\subsection{Basic control of the plot view\xlabel{basic_control}}

\subsection{Meta data\xlabel{meta_data}}

\subsection{Making a periodogram\xlabel{making_a_periodogram}}

\subsection{Folding time series data\xlabel{folding_data}}

\subsection{Fitting folded data\xlabel{fitting_data}}

%-----------------------------------------------------------------------
\section{Manipulating time series\xlabel{manipulating_time_series}}

\subsection{Combining individual series\xlabel{combing_time_series}}

\subsection{Detrending series\xlabel{detrending_time_series}}

\subsection{Basic arithmetic functions\xlabel{basic_arithmetic}}

%-----------------------------------------------------------------------
\section{More about periodograms\xlabel{more_periodograms}}


%-----------------------------------------------------------------------
\section{Fake data\xlabel{fake_data}}

\subsection{Faking it from scratch\xlabel{fake_from_scratch}}

\subsection{Faking it from existing data\xlabel{fake_from_data}}

%-----------------------------------------------------------------------
\section{Folding and binning data\xlabel{folding_and_binning}}

%-----------------------------------------------------------------------
\section{Fitting data\xlabel{fitting_data}}

%-----------------------------------------------------------------------
\section{Annotating your plots\xlabel{annotating your plots}}

\FROG\ shares a common infrastructure for graphics annotation with \SPLAT\

%-----------------------------------------------------------------------
\section{Web services\xlabel{web_services}}

\subsection{Displaying a time series using web serices\xlabel{display_a_time_series_via_ws}}

\subsection{Creating a Fourier Transform using web serices\xlabel{generating_an_fft_via_ws}}

%-----------------------------------------------------------------------
\section{Supported data formats\xlabel{supported_formats}}

\FROG\ can read and write time series and periodograms in two formats, a simple ASCII TEXT files or from a FITS file.


%-----------------------------------------------------------------------
\section{Acknowledgments\xlabel{acknowledgements}}

\FROG\ is written in
\htmladdnormallinkfoot{Java}{http://java.sun.com} by
\htmladdnormallinkfoot{SUN Microsystems Inc.}{http://www.sun.com/} and
contains software from:
\begin{itemize}
\item The JDOM Project (\htmladdnormallink{http://www.jdom.org/}
                        {http://www.jdom.org/}).
\item The Apache Software Foundation
      (\htmladdnormallink{http://www.apache.org/}{http://www.apache.org/} ).
\item The DIVA project
      (\htmladdnormallink{http://www.gigascale.org/diva/}
                         {http://www.gigascale.org/diva/}).
\item The Java Matrix Package (JAMA)
      (\htmladdnormallink{http://math.nist.gov/javanumerics/jama/}
                         {http://math.nist.gov/javanumerics/jama/}).
\end{itemize}

%-----------------------------------------------------------------------
\section{Changes\xlabel{changes}}
\subsection{This release, 0.7}
In this release \FROG\ has become part of the Starlink Java Collection and is now released under the GPL. The most significant changes in this release are:
\begin{itemize}

   \item The addition of a web service interface.
   
   \item The addition of a graphics annotation menu.
    
\end{itemize}

\end{document}
